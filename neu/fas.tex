\documentstyle[12pt,german]{report}
\pagestyle{empty}
\textwidth16.0cm 
\textheight24.0cm 
\oddsidemargin0.0cm 
\evensidemargin-.5cm 
\topmargin-0.0cm 
\renewcommand{\baselinestretch}{1.6}
\parskip1.5ex
\begin{document}
\parindent0pt
\baselineskip3.2ex

Das Studium der Physik habe ich an der RWTH Aachen absolviert und 1972 mit einer 
Diplomarbeit im Bereich der Elastizit"atstheorie abgeschlossen. 
Danach wechselte ich an die Universit"at Dortmund, wo ich 1975 mit 
einer Arbeit "uber den "`Einfluss von Kristallanisotropie und Phononen auf das 
Energiespektrum von Exzitonen`` promovierte. Im M"arz 1976 absolvierte ich einen 
ersten kurzen Forschungsaufenthalt in den USA am Physics Department der University 
of Illinois in Urbana. Von 1977 bis 1978 ging ich an das zentrale IBM 
Research Center nach Yorktown Heights, New York, um mich in ein neues 
zukunftstr"achtiges Arbeitsgebiet, die Oberfl"achen- und Grenzfl"achenphysik,
einzuarbeiten. Der Schwerpunkt der Arbeiten in Yorktown Heights bestand in der 
Entwicklung und numerischen Realisierung einer neuen und sehr effizienten
Methode zur theoretischen Beschreibung von Halbleiteroberfl"achen und 
Grenzfl"achen. Die Entwicklung dieser Methode und zahlreiche Anwendungen auf 
Halbleiter Oberfl"achen, Grenzfl"achen in Heterostrukturen und Supergittern 
sowie auf Defekte und Adsorbate an Oberfl"achen f"uhrten 1979 zu meiner 
Habilitation im Fach Physik an der Universit"at Dortmund. Im Juli und August 1980 
verbrachte ich einen weiteren Forschungsaufenthalt in den USA am Brooklyn College 
der City University of New York und am IBM Research Center in Yorktown Heights.  
Im Sommersemester 1983 und im Wintersemester 1983/84 nahm ich eine Lehrstuhlvertretung 
an der Universit"at Essen wahr. Seit 1986 habe ich eine C4-Professor f"ur Theoretische 
Physik im Institut f"ur Festk"orpertheorie inne. Hauptarbeitsgebiet ist die 
{\it ab-initio} Theorie elektronischer, struktureller, vibronischer,
chemischer und optischer Eigenschaften von Volumenhalbleitern, Oberfl"achen und 
Grenzfl"achen, die wir unter Verwendung modernster Methoden aus der  
Quantenmechanik, Vielteilchentheorie, Statistischen Physik und  
Computer-Mo\-del\-lie\-rung studieren. In der aktuellen Festk"orperphysik sind 
Oberfl"achen und Grenzfl"achen von Halbleitern, Metallen und Isolatoren von 
zentraler Bedeutung. Infolge der reduzierten Dimensionalit"at geben sie zu einer 
F"ulle grundlegend neuer Ph"anomene Anlass und sind z.B. f"ur Halbleiterbauelemente 
in der Mikro-, Nano- und Optoelektronik sowie f"ur die Katalyse und die Korrosion 
von gro"ser technologischer Relevanz. Unsere Arbeiten werden in Kooperation mit 
Forschern im In- und Ausland durchgef"uhrt.

Neben den Aufgaben in Lehre und Forschung habe ich auch in vielf"altiger Weise 
Gutachteraufgaben (zahlreiche wissenschaftliche Zeitschriften, DFG, BMBF, F"orderung 
der wissenschaftlichen Forschung in "Osterreich, Fritz-Haber Institut der MPG, 
bayerischer Forschungsverbund FOROPTO, Auswahlgremium f"ur die Grundf"orderung 
des Cusanuswerkes) und Aufgaben in der akademischen Selbstverwaltung wahrgenommen. 
Von 1987 bis 1992 war ich Mitglied im Wissenschaftlichen Beirat des 
Universit"atsrechenzentrums und der Senatskommission f"ur Allgemeine Datenverarbeitung, 
deren Vorsitz ich von 1990 bis 1992 inne hatte. 
Weiterhin war ich Mitglied in einer Reihe von Kommissionen und Aussch"ussen  
des Fachbereichs und langj"ahriges Mitglied im Fachbereichsrat.  
Von 1997 bis 2002 war ich dar"uber hinaus Dekan des Fachbereichs Physik. 
Daneben bin ich seit vielen Jahren Mitglied im International Advisory Board der 
Zeitschrift "`Surface Science`` und Mitherausgeber des "`European Journal of 
Physics``.

\end{document}

