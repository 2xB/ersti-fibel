\documentclass[12pt,a4paper,twocolumn]{report}
\usepackage[latin1]{inputenc}
\usepackage{ngerman}
\begin{document}
\chapter*{Radio Q 90.9}

Seit nun mehr 5 Jahren gibt es nun M�nsters Hochschulradio Q 90.9. Von Montag bis Freitag geht der Sender von Stundenten f�r Studenten live auf Sendung. Knackig, frech und jung, informativ und inhaltsreich - so pr�sentiert sich der gr��te Hochschulsender Nordrhein-Westfalens. Themen rund um das M�nsteraner Campusleben werden auf dem Sendeplan nat�rlich gro� geschrieben. Ob Studiengeb�hren oder Lifestyle, ob "`Prof-Portrait"' oder Mensa-Plan, Wohnungsnot oder Wissenschaftspolitik - das erkl�rte Ziel von Q 90.9 ist es, neben den Bereichen Wissenschaft und Hochschule auch Kultur, Sport, Freizeit und Unterhaltung einen Platz im Programm zu bieten. Schlie�lich findet studentisches Leben nicht allein in Seminaren und Vorlesungen statt.

Das geeignete Forum dazu bieten 35 Stunden Sendezeit, die Q 90.9 w�chentlich mit Wissenswertem und Unterhaltendem f�llt. Aus dem Bett holt das Hochschulradio seine H�rer schon um 8 Uhr mit dem morgendlichen Coffeshop, in dem es Aktuelles aus Hochschule, Wissenschaft, Kultur, Lifestyle, Musik und Sport gibt.

Zur Mittagszeit werdet ihr nat�rlich auch mit den aktuellen Informationen �ber M�nsters Mensen, sprich dem Mensaplan, gef�ttert.

Nachmittags von 14 bis 15 Uhr wird auf den Deutschlandfunk umgeschaltet, der euch ab ca. 14:35 Uhr mit aktuellen Informationen zu "`Campus \& Karriere"' versorgt, es werden Fragen wie "`Was kann man mit dem neuen Studiengang an der HU Berlin werden?"' und "`Was ist ein Hochschulrahmengesetz"' beantwortet.

Von 16 bis 17 Uhr wird wiederum auf einen anderen Sender umgeschaltet, zu dieser Zeit wechselt Radio Q auf WDR 5 "`Leonardo"' auf dem Wissenschaftliche Themen wie Doppelhelix  oder auch die Dinosaurier behandelt werden. Zust�tzlich gibt es den Service Gesundheit, Buchtipps und die Kleine Anfrage, die sich den R�tseln des Alltags widmet.

Ab 18 Uhr geht es dann wieder Live weiter mit dem "`Abwasch"', in dem ihr �ber das Tagesgeschehen auf dem Laufenden gehalten werdet und euch auf den Abend in M�nster vorbereitet. Danach geht es jeden Abend anders weiter, ob es jetzt die "`Campuscharts"' am Montagabend sind oder doch lieber "`Das Sofa"', in dem mit Profs und Promis gesprochen wird, am Dienstagabend ist, jeder Abend hat ein anderes, eigenes Programm.

Musikalisch ist Radio Q auch anders, es werden nicht die Hits gespielt, die euch �berall um die Ohren gehauen werden, nein, man legt mehr Wert auf andere Musik: Unter dem Motto "`Q rockt"' gibt es aus den meisten Musikrichtungen etwas, neben Rock, Alternative oder Indie-Pop werden auch Songs aus den Bereichen Hip-Hop, Reggae, Dance, Pop, Drum'n'Bass, Jazz und Soul gespielt. Radio Q versucht sein Musikprogramm mit Herz und Seele zu machen.

Radio Q stellt euch auch zahlreiche Bands vor und gibt aktuelle Daten zu Ver�ffentlichungen und Tourneen aus. So vielf�ltig wie die Berichterstattung sind auch die Aufgaben der ausschlielich ehrenamtlich arbeitenden Mitarbeiter. "`Neben der klassischen journalistischen Redaktionsarbeit haben wir Mitarbeiter in den Bereichen Marketing, �ffentlichkeitsarbeit, Ausbildung, Technik und Internet"', erkl�rt Martin Adler, Chefredakteur Wort. Im Mittelpunkt steht aber immer noch das "`Radiomachen"': Beitr�ge produzieren, Nachrichten live einsprechen, eine Sendung moderieren oder die Musikredaktion unterst�tzen. Es gibt eine Menge zu tun, organisatorisch wie redaktionell.

Derzeit sammelt Radio Q Unterschriften, damit sie ab dem n�chsten Sommersemester 20 Cent vom Semesterbeitrag bekommen. Diese 20 Cent von jedem Studenten werden dann in neue Technik, z.B. neue Sendeanlage mit Antenne, PCs f�r die Redaktion und �hnliches gesteckt. Des Weiteren wird das Geld auch zur Bezahlung von Telefonrechnungen und dem technischen Verschlei� und andere Erneuerungen genutzt.

Mittwochs um 16 Uhr findet bei Radio Q in der Bismarckallee 3, direkt neben der Mensa 1 am Aasee, eine Sprechstunde statt, in der ihr alles �ber Radio Q und wie ihr mitmachen k�nnt erfahrt. Radio Q sucht immer Studenten, die mitmachen. Alle Mitarbeiter machen das ehrenamtlich. Vorraussetzung zum Mitmachen ist, dass ihr Student an einer von M�nsters Hochschulen seid. Nicht nur Germanisten und Kommunikationswissenschaftler sind gefragt.
Q 90.9 ist schlie�lich kein reines Uniradio, sondern ein Sender f�r alle f�nf Hochschulen der Stadt. Die "`\textbf{Q}uinti Campi"' standen auch bei der Namensgebung von Radio \textbf{Q} Pate. Bei der Gestaltung der Programminhalte m�chten die Radiomacher gerne das Know-How s�mtlicher Fachbereiche einbeziehen. Experten von der Arch�ologie bis zur Wirtschaftsinformatik mit Interesse am Medium Radio sind bei Q 90.9 herzlich willkommen.
Interessierte k�nnen einfach mittwochs um 16 Uhr bei der Sprechstunde vorbeischauen, weitere Infos unter (02 51) 83-7 90 90 einholen oder eine Mail an ausbildung@radioq.de schicken. Im Internet ist Q 90.9 unter www.radioq.de vertreten.

\begin{flushright}
(Matthias Dropmann)
\end{flushright} 
\end{document}