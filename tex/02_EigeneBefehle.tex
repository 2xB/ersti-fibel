% Variablen für Titelseite/Seitenvorlage
% Jahr und Semester gegebenenfalls ändern!
\newcommand{\fibeltitel}{Erstsemester-Informationszeitung des Fachbereichs Physik der Westfälischen Wilhelms-Universität Münster}
% XXX Jedes Jahr aktualisieren!
\newcommand{\fibeljahr}{2017/2018}
\newcommand{\fibelsemester}{Wintersemester}
\newcommand{\fibelautor}{Fachschaft Physik der WWU Münster; Simon May, Benedikt Bieringer}

% Befehl \email: E-Mail-Adresse mit mailto-Link (anklickbar in der PDF-Datei)
% 	Parameter #1: Gewünschte E-Mail-Adresse
\newcommand{\email}[1]{\href{mailto:#1}{\texttt{#1}}}

% Befehl \hrefurl: Link, der als URL gesetzt wird, wo aber Verweisziel und Text
% 	nicht identisch sind
% 	Parameter: Wie bei \href
\newcommand{\hrefurl}[2]{\href{#1}{\texttt{#2}}}

% Befehl \fibelsig: Für Unterschriften unter Artikeln
% 	Parameter #1: Name des Autors
\newcommand{\fibelsig}[1]{\par\hfill(#1)\par}

% Befehl \fibelimgtext: Bild mit zusätzlichem Text (z.B. Quellen-URL)
% 	Parameter #1 (optional): Position/Ausrichtung des Textes
% 		(default: unten rechts)
% 	Parameter #2: Bild (z.B. \includegraphics)
% 	Parameter #3: Zusätzlicher Text (standardmäßig \scriptsize)
\NewDocumentCommand{\fibelimgtext}{O{bottom right} >{\TrimSpaces}m m}{%
	\def\fibrotate{90}%
	\def\fibxsep{2pt}%
	\def\fibysep{0}%
	\def\fibtxt{#3}%
	\def\fibalign{l}%
	\def\fibianchor{south east}%
	\def\fibtanchor{south west}%
	\ifstr{#1}{top left}{%
		\def\fibalign{r}%
		\def\fibianchor{north west}%
		\def\fibtanchor{north east}%
	}{\ifstr{#1}{bottom left}{%
		\def\fibalign{r}%
		\def\fibianchor{south west}%
		\def\fibtanchor{south east}%
	}{\ifstr{#1}{top right}{%
		\def\fibianchor{north east}%
		\def\fibtanchor{north west}%
	}{\ifstr{#1}{bottom right}{%
		% bottom right is default
	}{\ifstr{#1}{above left}{%
		\def\fibianchor{north west}%
		\def\fibtanchor{south west}%
		\def\fibrotate{0}%
		\def\fibxsep{0}%
		% Texthöhe als 0 erscheinen lassen
		\def\fibtxt{\raisebox{0pt}[0pt][0pt]{#3}}%
	}{\ifstr{#1}{above right}{%
		\def\fibianchor{north east}%
		\def\fibtanchor{south east}%
		\def\fibrotate{0}%
		\def\fibxsep{0}%
		% Texthöhe als 0 erscheinen lassen
		\def\fibtxt{\raisebox{0pt}[0pt][0pt]{#3}}%
	}{\ifstr{#1}{below left}{%
		\def\fibianchor{south west}%
		\def\fibtanchor{north west}%
		\def\fibrotate{0}%
		\def\fibxsep{0}%
		\def\fibysep{2pt}%
		% Texthöhe als 0 erscheinen lassen
		\def\fibtxt{\raisebox{-\totalheight}[0pt][0pt]{#3}}%
	}{\ifstr{#1}{below right}{%
		\def\fibianchor{south east}%
		\def\fibtanchor{north east}%
		\def\fibrotate{0}%
		\def\fibxsep{0}%
		\def\fibysep{2pt}%
		% Texthöhe als 0 erscheinen lassen
		\def\fibtxt{\raisebox{-\totalheight}[0pt][0pt]{#3}}%
	}{%
		\makeatletter%
		\fibelwarning{Ungueltige Option fuer den Befehl \string\fibelimgtext: #1}%
		\makeatother%
	}}}}}}}}%
	\makebox[\widthof{#2}][\fibalign]{%
		\begin{tikzpicture}
			\node[anchor=\fibianchor, inner sep=0] at (0, 0)
			{#2};
			\node[anchor=\fibtanchor, inner ysep=\fibysep, inner xsep=\fibxsep] at (0, 0)
			{\rotatebox{\fibrotate}{\scriptsize\fibtxt}};
		\end{tikzpicture}%
	}%
}

% Befehl \fibelwerbung: Für eine Seite mit Werbung
% 	Stern: Verwende \thispagestyle{empty} statt \thispagestyle{scrplain}
% 		(Umschlagsseite)
% 	Optionaler Parameter #1: Optionen für \fibfullpage
% 	Optionaler Parameter #2: Erstellt eine Seite mit dem Parameter als Inhalt;
%	falls leer: Verwendet das Wort „Werbung“ in groß und zentriert
\NewDocumentCommand{\fibelwerbung}{s >{\TrimSpaces}O{\Huge\bfseries Werbung} >{\TrimSpaces}o}{%
	\IfBooleanTF{#1}{\thispagestyle{empty}}{\thispagestyle{scrplain}}
	\IfValueTF{#3}{%
		\fibfullpage[#2]{#3}
	}{%
		\fibfullpage{#2}
	}
	% irgendein Output wird benötigt, damit überhaupt eine neue Seite
	% angefangen wird
	\hspace{0pt}
	\clearpage
}

% Befehl \fibfullpage: Erzeugt den Inhalt als Seitenhintergrund
% 	Optionaler Parameter #1: Optionen für die TikZ-\node
% 	Parameter #2: Inhalt
\makeatletter
\NewDocumentCommand{\fibfullpage}{O{} m}{%
	\def\fibfpcontent{%
		\begin{tikzpicture}[remember picture, overlay]
		\node[inner sep=0, outer sep=0, #1] at (current page) {#2};
		\end{tikzpicture}%
		
	}%
	% falls das „crop“-Paket geladen ist, muss der Hintergrund auf eine andere
	% Weise erzeugt werden, da dieser sonst ÜBER die Schnittmarken gesetzt wird
	\ltx@ifpackageloaded{crop}{%
		\global\let\CROP@user@a\fibfpcontent%
		\gdef\CROP@user@b{\global\let\CROP@user@a\relax}
	}{%
		\AddThispageHook{\fibfpcontent}
	}%
}
\makeatother

% Zähler für Umgebung fibelurl (wird mit jeder \section zurückgesetzt)
\newcounter{fibelurlctr}[section]
\crefformat{fibelurlctr}{#2[#1]#3}

% Umgebung „fibelurl“: Für nummerierte Linklisten mit Verweisen wie z.B. im
% Artikel Geld/BAföG
\newenvironment{fibelurl}
{\refstepcounter{fibelurlctr}[\arabic{fibelurlctr}]~}
{\par}

% Befehl \pullquotenl: Leerzeile für die „pullquote“-Umgebung (Bild in der
% Mitte, von Text umflossen)
\newcommand{\pullquotenl}{\par~}
% Befehl \fibelsigpq: Für Unterschriften unter Artikeln mit pullquote (ohne
% \par)
% 	Parameter #1: Name des Autors
\newcommand{\fibelsigpq}[1]{\hfill(#1)}

%%%%%%%%%%%%%%%%%%%%%%%%%%%%%%
% !!! Die Befehle ab hier sind nicht zur direkten Verwendung durch den gemeinen
% !!! Fibel-Schreiberling gedacht ;-)

% Temporäre Variablen
\newlength{\fibtemp}
\newcounter{fibtemp}

% Definition der Linie in der Kopfzeile
\NewDocumentCommand{\fibelheadsepline}{O{even}}
{\vspace{-0.6cm}%
\ifstr{#1}{even}{\Huge$\Phi$\hfill}{}%
\rule[2.6mm]{0.96\linewidth}{0.5mm}%
\ifstr{#1}{odd}{\hfill\Huge$\Phi$}{}}

% Definition des Textes an den Seitenrändern jeder Seite
\newcommand{\fibelmargin}{\Large Ersti-$\Phi$bel \fibeljahr}

% Erzeugung einer Warnung
\newcommand{\fibelwarning}[1]{\GenericWarning{}{LaTeX warning: #1}}
