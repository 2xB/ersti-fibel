\newcommand{\fibelvp}[2]{
	\minipage{0.24\textwidth}
		\begin{centering}
			\includegraphics[width=\linewidth]{#1}
			\textbf{#2}
		\end{centering}
	\endminipage\hfill
}

\section{Organisatorisches zur O-Woche}
% Schriftgröße für \subsection in diesem Teil anpassen
\addtokomafont{subsection}{\normalsize}
\textbf{Um den Überblick wiederzufinden, wenn man ihn mal verloren hat.}

Hey! Unser selbstgestecktes Ziel mit der O-Woche ist, euch in einer unterhaltsamen und informativen Woche einen
Einstieg in euer Studileben mit dem an die Hand zu geben, was erfahrungsgemäß nützlich ist, um den Sprung ins kalte Wasser
zu meistern. Dazu gehören sowohl lehrreiche Vorträge wie auch das Kennenlernen eurer Kommilitonen und vielleicht auch
das Knüpfen erster Freundschaften. Nun steht hinter der O-Woche nicht nur das Koordinationsteam, sondern es packt jeder
aus Fachschaft und Umgebung so gut es geht an. Da wir das Ganze für euch machen, liegt es uns natürlich sehr am Herzen,
dass die Veranstaltungen unserer O-Woche für euch ein schönes Erlebnis sind. Dafür seien folgende Adressen
besonders hervorgehoben:

\subsection{Der Fachschaftsrat Physik}
\paragraph{Mail}
Alle Mails, die an \email{fsphys@uni-muenster.de} gehen, erreichen die zuständigen Verantwortlichen aus dem
O-Wochen-Team. Es sei hier darauf hingewiesen, dass Mails eine Form der asynchronen Kommunikation darstellen,
auch wenn wir uns um eine schnelle Antwort bemühen.

\paragraph{Fon/Fachschaftsraum}
Fast die gesamte O-Woche über findet man Fachschaftler in der direkten Nähe des Fachschaftsraumes -- sei es, weil
dort eine Station eines Spieles stattfindet, gerade etwas für den nächsten Tag vorbereitet oder sich mal eine kurze
Verschnaufpause gegönnt wird. Wenn ihr in der Nähe seid und eine Frage zur O-Woche oder zum Studium allgemein habt,
kommt doch mal rum, im Idealfall können wir sofort helfen. Ansonsten könnt ihr Leute, die direkt im Fachschaftsraum
sind, auch telefonisch erreichen: +49~251~83-34985.

\subsection{Vertrauenspersonen}
Wir wissen, wie wichtig es ist, jemanden zu Reden zu haben -- sei es über die O-Woche selbst, die
allgemeine Studienplanung oder eine schwere emotionale Situation, die ihr gerade durchmacht. Genau aus dem Grund
gibt es für den Zeitraum der gesamten O-Woche die folgenden vier Vertrauenspersonen, mit denen ihr im Vertrauen über
alles sprechen könnt. Sprecht uns an -- ihr erkennt uns an einer Kapitänsbinde!

\fibelvp{res/vorstellungsfotos/simon_may_alexandra_everwand_edited_cropped.jpg}{Alexandra (r.)}
\fibelvp{res/vorstellungsfotos/andrea_garner_cropped.JPG}{Andrea}
\fibelvp{res/vorstellungsfotos/benedikt_bieringer.png}{Benedikt B.}
\fibelvp{res/vorstellungsfotos/fernando_romahn.png}{Fernando}
\fibelsig{alphabetisch geordnet}
