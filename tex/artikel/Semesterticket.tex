
\subsection{Das Kultursemesterticket}
Seit es 2015 durch eine Urabstimmung in der Studierendenschaft eingeführt wurde gibt es neben dem Semesterticket das sogenannte Kultursemesterticket. Dieses wird ähnlich wie das Semesterticket aus den Semesterbeiträgen finanziert und erlaubt Studierenden viele kulturelle Einrichtungen in Münster vergünstigt oder sogar kostenlos zu besuchen. Dies handelt der AstA jeweild direkt mit der ensprechenden Einrichtung aus und der Grad der Vergünstigung kann daher stark variieren. So bieten z.\,B. das Museum für Lackkunst und der Literaturverein freien Eintritt,  während das Theater Münster ein festes Kontingent an Freikarten und kostenlose Restplätze anbietet und das GOP Variet\'e für einige Veranstaltungstermine jede Woche Sonderpreise für Studierende hat. Fußballfans dürfte interessieren, das der AStA für jedes Heimspiel des Sc Preussen Münster 50 Freikarten bekommt, die man sich beim AStA sichern kann.
Die vollständige Übersicht gibt es unter
\vspace{-1ex}
\begin{center}
	\url{https://www.asta.ms/kultursemesterticket}
\end{center}

\smallskip
