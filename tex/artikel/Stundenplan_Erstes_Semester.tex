\section[Stundenplan 1.~Semester]{Dein Stundenplan im 1.~Semester}
\vspace{-0.5cm}
\subsubsection*{für 1-Fach-Bachelor-Physiker, 2-Fach-Bachelor-Physiker und Geophysiker}
\begin{minipage}{\textwidth}
% Keine Trennlinie vor Fußnoten in dieser minipage
\setfootnoterule{0cm}
% Die Länge \fibtemp ist die Breite einer Spalte in der Tabelle
% (außer Spalte mit den Zeiten)
\setlength{\fibtemp}{0.152\textwidth}
% Der Befehl \fibnl ist ein Zeilenumbruch
\let\fibnl=\par

\centering
% Verlinkung von Fußnoten im PDF klappt nicht mit tabularx :-(
% geht mit tabular, tabular*
\begin{tabular}{| >{\footnotesize}p{0.1\textwidth} | *{5}{>{\footnotesize\centering\arraybackslash}p{\fibtemp}|}}
\hline
	\textbf{Uhrzeit} &
	\textbf{Montag} &
	\textbf{Dienstag} &
	\textbf{Mittwoch} &
	\textbf{Donnerstag} &
	\textbf{Freitag}
\\ \hline
08:00--\fibnl
10:00 Uhr &
	\textbf{Physik~I\fibnl
		Übung}\fibnl
	(diverse Seminarräume) &
	\textbf{Mathe für Physiker~I\footnote{Nicht für 2-Fach-Bachelor-Studenten.\label{stundenplan:mfp1}} Übung}\fibnl
	(diverse Seminarräume) &
	\textbf{Physik~I Vorlesung}\fibnl
	HS~1 &
	\textbf{Physik~I\fibnl
		Übung}\fibnl
	(diverse Seminarräume) &
	Informatik~I\cref{stundenplan:informatik} Übung\fibnl
	(diverse Seminarräume)
\\ \hline
10:00--\fibnl
12:00 Uhr &
	\textbf{Mathe für Physiker~I\cref{stundenplan:mfp1} Vorlesung}\fibnl
	HS~2 &
	\textbf{Physik~I Vorlesung}\fibnl HS~1 &
	&
	\textbf{Mathe für Physiker~I\cref{stundenplan:mfp1} Vorlesung}\fibnl
	HS~2 &
	\textbf{Physik~I Vorlesung}\fibnl
	HS~1
\\ \hline
12:00--\fibnl
13:00 Uhr &
	Chemie\footnote{Für 1-Fach-Bachelor Physik mit dem Modul Chemie als fachübergreifende Studie.
	
	Die Termine zu den Übungen werden in der Vorlesung verteilt.\label{stundenplan:chemie}} Vorlesung\fibnl
	C1 &
	Chemie\cref{stundenplan:chemie} Vorlesung\fibnl
	C1 \flushright
	\textbf{\& REP (KP~304)}
	&
	Chemie\cref{stundenplan:chemie} Vorlesung\fibnl
	C1 \flushright
	\textbf{\& TUT (KP~404)}
	&
	Chemie\cref{stundenplan:chemie} Vorlesung\fibnl
	C1 &
\\ \hdashline
13--14 Uhr &
	& & & &
\\ \hline
14:00--\fibnl
16:00 Uhr &
	Informatik~I\footnote{Für 1-Fach-Bachelor Physik mit dem Modul Informatik als fachübergreifende Studie.
	
	Die Termine zu den Übungen werden in der Vorlesung verteilt.\label{stundenplan:informatik}} Vorlesung\fibnl
	M~1 &
	&
	Einführung in die Geophysik\footnote{Für 1-Fach-Bachelor Geophysik und 1-Fach-Bachelor Physik mit dem Modul Geophysik als fachübergreifende Studie.
	
	Die Termine zu den Übungen werden in der Vorlesung verteilt.\label{stundenplan:geophysik}} Vorlesung\fibnl
	HS~AP &
	Informatik~I\cref{stundenplan:informatik} Vorlesung\fibnl
	M~1 &
\\ \hline
16:00--\fibnl
18:00 Uhr &
	& &
	\textbf{REP (KP~104)}
	& &
\\ \hline
18:00--\fibnl
20:00 Uhr &
	&
	&
	Fachschafts-Sitzung\fibnl
	(Ihr seid alle willkommen!) &
	&
\\ \hline
\end{tabular}
\vspace{-1ex}
\end{minipage}
{\footnotesize
\textbf{REP} Mathe-Repetitorium: \textit{zweistündiges, freies Angebot zum Vertiefen grundlegender Mathematik mit wöchentl.\ Themen}\\
\textbf{TUT} Tutorium: \textit{zweistündiges, freies Angebot zum Vertiefen allgemein grundlegender Themen}
}

{\small
Des Weiteren können 1-Fach-Bachelor-Studenten, die weder Chemie, Geophysik, noch Informatik als fachübergreifende Studie wählen möchten, aus folgenden vorgefertigten Modulen als fachübergreifende Studien wählen:
\begin{itemize}[nosep]
	\item Philosophie für Physiker
	\item Mathematik
	\item Theoretische Grundlagen der Psychologie
	\item Einführung in die Betriebswirtschaftslehre
	\item Einführung in die Volkswirtschaftslehre
	\item Spanisch für Naturwissenschaftler
	\item (Deutsch als Fremdsprache)
\end{itemize}
oder ein selbst zusammengestelltes fachübergreifendes Modul, bei dem der Zusammenhang zur Physik erkennbar ist, oder das zur Berufsfähigkeit dient, bestimmen.
Dies ist aber nur nach Absprache mit dem Studiendekan möglich (aktuell: Prof.\ Dr.\ Tilmann Kuhn, Institut für Festkörpertheorie~(FT), Raum~706).
Die genauen Vorlesungs-/Übungszeiten der zusätzlichen fachübergreifenden Studien erfahrt ihr in der O-Woche von der Fachschaft Physik, direkt von der jeweiligen Fachschaft des zuständigen Fachbereiches, oder ihr schaut einfach im Vorlesungsverzeichnis (HIS~LSF, \url{https://studium.uni-muenster.de/qisserver/rds?state=wtree&search=1}) der Uni nach.
}
