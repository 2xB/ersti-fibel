% XXX Jedes Jahr O-Wochen-Plan aktualisieren!
% Zeilenumbruch \\ als \fibnlx zwischenspeichern (in Tabellen wird \\ für eine
% neue Tabellenzeile umdefiniert)
\let\fibnlx=\\
% Befehl \fibnl ist ein Zeilenumbruch mit etwas Freiraum darunter
\DeclareDocumentCommand{\fibnl}{}{\fibnlx[\baselineskip]}

% Länge \fibprogrammcw ist die Breite einer Spalte in der Programmtabelle
% (außer Spalte mit den Zeiten)
\newlength{\fibprogrammcw}
\setlength{\fibprogrammcw}{0.214\textheight}

\begin{landscape}
\section{Programm der Physik-Orientierungseinheit (O-Woche/Ersti-Woche)}
\renewcommand{\arraystretch}{1.8}
\footnotesize
\begin{tabular}{| >{\bfseries\hfill}p{0.08\textheight} | *{4}{p{\fibprogrammcw} |}}
\hline
Uhrzeit &
	\textbf{Montag, 10.10.} &
	\textbf{Dienstag, 11.10.} &
	\textbf{Mittwoch, 12.10.} &
	\textbf{Donnerstag, 13.10.}
\\ \hline
10:00\vspace{\baselineskip} &
	\multirow{4}{\fibprogrammcw}[-2mm]{%
		\textbf{Einführungsveranstaltung}\fibnl
		\hspace*{\fill}
		\textit{Hörsaal~AP}\fibnl
		\textbf{Tutorien und Institutsführung}\fibnl
		\textbf{Mittagessen}
	} &
	\multirow{2}[2]{\fibprogrammcw}[-3mm]{%
		\textbf{Infoveranstaltung~II}\fibnlx
		(Gremien, IVV~NWZ+ZIV, BAföG)\fibnl
		\hspace*{\fill}
		\textit{Hörsaal~AP}} &
	\multirow{2}[6]{\fibprogrammcw}[-3mm]{%
		\textbf{Ausweichtermin Infoveranstaltung~I}\fibnlx
		(nur für Zwei-Fach-Bachelor)\fibnl
		\hspace*{\fill}
		\textit{Hörsaal KP~404}} &
\\ \cline{1-1}
11:00\vspace{\baselineskip} & & & &
\\ \cline{1-1}\cline{3-5}
12:00 & &
	Mittagspause &
	\textbf{Laborführungen}\fibnl
		\hspace*{\fill}
		\textit{Treffen vor der Fachschaft} &
	\multirow{2}[20]{\fibprogrammcw}{\textbf{Infoveranstaltung~III}\fibnlx
		(jDPG \& andere studentische Gruppierungen)\fibnl
		\textbf{Plenum \& Preisverleihung}\fibnl
		\hspace*{\fill}
		\textit{Hörsaal~1}}
\\ \cline{1-1}\cline{4-4}
13:00\vspace{2\baselineskip} & & &
	Mittagspause &
\\ \hline
14:00 &
	\multirow{2}[12]{\fibprogrammcw}{\textbf{Infoveranstaltung~I}\fibnlx
		(Bachelor Physik/Geophysik, Zwei-Fach-Bachelor)\fibnl
		\hspace*{\fill}
		\textit{Hörsaal~AP}} &
	\textbf{Buch-Club}\fibnlx
	(Fachliteratur)\fibnl
	\hspace*{\fill}\textit{Hörsaal~AP} &
	\multirow{4}[38]{\fibprogrammcw}{\textbf{Stadtspiel}\fibnl
		\hspace*{\fill}
		\textit{Treffen an der Freitreppe}\fibnl~\fibnl~\fibnl~\fibnl~\fibnl
		anschließend Grillen vor der Fachschaft\fibnlx
		(bei passender Wetterlage)} &
	\multirow{2}[8]{\fibprogrammcw}{%
		\textbf{"Kaffeetrinken" mit den Professoren}\fibnl
		\hspace*{\fill}
		\textit{Foyer IG1}
	}
\\ \cline{1-1}\cline{3-3}
15:00 &
	&
	\textbf{Vortrag der Polizei}\fibnlx
		inkl.\ Fahrradregistrierung\fibnl
		\hspace*{\fill}\textit{Hörsaal~AP} &
	&
\\ \cline{1-3}\cline{5-5}
16:00 &
	\multirow{2}[4]{\fibprogrammcw}{\textbf{Physikspiel}\fibnl
		\hspace*{\fill}
		\textit{Treffen an der Freitreppe}} &
	\textbf{Ersti-Begrüßung des Rektorats}\fibnl
		\hspace*{\fill}
		\textit{Foyer im Schloss}
	& &
\\ \cline{1-1}\cline{3-3}\cline{5-5}
17:00 & &
	\multirow{2}{\fibprogrammcw}{%
		\textbf{Konstruktionswettbewerb}\fibnl
		\hspace*{\fill}
		\textit{Treffen im Foyer IG1}
	}
	& &
	\multirow{3}[4]{\fibprogrammcw}{%
		\textbf{Spieleabend (\emph{ab} 17:00~Uhr)}\fibnlx
		Brettspiele, Kartenspiele, Pen~\&~Paper und mehr!\fibnl
		\hspace*{\fill}
		\textit{Seminarraum KP~104}
	}
\\ \cline{1-2}
18:00 &
	\multirow{2}{\fibprogrammcw}{%
		\textbf{Grillen}\fibnlx
		(bei passender Wetterlage)
	}
	& & &
\\ \cline{1-1}\cline{3-3}
abends &
	&
	\textbf{Vorabendprogramm\fibnlx
		(Pizza Essen, "Vorglühen", \dots)}\fibnlx
	\hspace*{\fill}
	\textit{Hörsaal~AP}\fibnl
	\textbf{Kneipenbingo (20:00 Uhr!)}\fibnlx
		\hspace*{\fill}
		\textit{Treffen vor der Cavete} &
	&
\\ \hline
\end{tabular}

\smallskip

% XXX Jedes Jahr Astroseminar-Termin aktualisieren!
\textbf{Du findest Astrophysik spannend?
	Zum Thema "Auf den Wogen von Raum und Zeit" findet dieses Jahr wieder das jährliche Astroseminar am Fr.\ und Sa., 21.10. \& 22.10. (Woche nach der Ersti-Woche) statt.
	Eintritt frei -- Anmeldung erforderlich!
	Mehr Infos: \url{https://www.uni-muenster.de/Astroseminar}}
\end{landscape}
