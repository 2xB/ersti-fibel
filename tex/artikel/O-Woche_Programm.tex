% XXX Jedes Jahr O-Wochen-Plan aktualisieren!
% Zeilenumbruch \\ als \fibnlx zwischenspeichern (in Tabellen wird \\ für eine
% neue Tabellenzeile umdefiniert)
\let\fibnlx=\\
% Befehl \fibnl ist ein Zeilenumbruch mit etwas Freiraum darunter
\DeclareDocumentCommand{\fibnl}{}{\fibnlx[\baselineskip]}
\DeclareDocumentCommand{\fibabstand}{}{\vspace{.5\baselineskip} }

% XXX Relevanz dieser Definition prüfen
% Nur für den Dienstags-Einschub.
\newlength{\fibprogrammeinschub}
\setlength{\fibprogrammeinschub}{0.005\textheight}

% Länge \fibprogrammcw ist die Breite einer Spalte in der Programmtabelle
% (außer Spalte mit den Zeiten)
\newlength{\fibprogrammcw}
\setlength{\fibprogrammcw}{0.214\textheight - 0.25\fibprogrammeinschub}


\begin{landscape}
\section{Programm der Physik-Orientierungseinheit (O-Woche/Ersti-Woche)}
\renewcommand{\arraystretch}{1.8}
\footnotesize
\begin{tabular}{
	|
	>{\bfseries\hfill} % füge „\bfseries\hfill“ zu Beginn jeder Zelle dieser Spalte ein (also fett und rechtsbündig)
	p{0.08\textheight} % Textspalte (mehrzeilig) der Breite 0.08\textheight
	|
	*{2} % füge die folgende Definition 2x ein
	{
		p{\fibprogrammcw} % Textspalte (mehrzeilig) der Breite \fibprogrammcw
		|
	}
	% Dienstags-Einschub.
	p{\fibprogrammeinschub} % Textspalte (mehrzeilig) der Breite \fibprogrammeinschub
	|
	*{2} % füge die folgende Definition 2x ein
	{
		p{\fibprogrammcw} % Textspalte (mehrzeilig) der Breite \fibprogrammcw
		|
	}
}
\hline
%%
%% TABLE HEADER
%%
Uhrzeit &
	\textbf{Montag, 01.10.} &
	\textbf{Dienstag, 02.10.} &
	\multirow{10}{*}{
		\hspace*{-2mm}\rotatebox{-90}{\hspace*{.07cm} \textbf{Mittwoch} ab 13 Uhr\textbf{: Gemütliches Beisammensein in kleinerem Rahmen}}
	}&
	\textbf{Donnerstag, 04.10.} &
	\textbf{Freitag, 05.10.}
%% Horizontal Lines
\\ \cline{1-3}\cline{5-6}
%%
%%  10:00
%%
10:00\fibabstand\fibabstand\fibabstand &
% Mo
	\multirow{4}{\fibprogrammcw}[-2mm]{%
		\textbf{Einführungsveranstaltung}\fibnl
		\hspace*{\fill}
		\textit{Hörsaal~AP}\fibnl
		\textbf{Tutorien und Institutsführung}\fibnl
		\textbf{Mittagessen}
	} & 
% Di & Mi
	\multirow{2}[2]{\fibprogrammcw}[-3mm]{%
		\textbf{Infoveranstaltung~II}\fibnlx
		(Gremien, IVV~NWZ+ZIV, BAföG)\fibnl
		\hspace*{\fill}
		\textit{Hörsaal~AP}} & &
% Do
	\multirow{2}[6]{\fibprogrammcw}[-3mm]{%
		\textbf{Ausweichtermin Infoveranstaltung~I}\fibnlx
		(nur für Zwei-Fach-Bachelor)\fibnl
		\hspace*{\fill}
		\textit{Hörsaal KP~404}} &
% Fr
	\textbf{Vortrag der Polizei}\fibnlx
		inkl.\ Fahrradregistrierung\fibnl
		\hspace*{\fill}\textit{Hörsaal~AP}
%% Horizontal Lines
\\ \cline{1-1} \cline{6-6}
%%
%%  11:00
%% Mo & Di & Mi & Do
11:00 \fibabstand & & & & &
% Fr
	\multirow{2}[15]{\fibprogrammcw}{\textbf{Infoveranstaltung~III}\fibnlx
		(jDPG \& andere studentische Gruppierungen)\fibnlx[0.5em]
		\textbf{Plenum \& Preisverleihung}\fibnl
		\hspace*{\fill}
		\textit{Hörsaal~1}}
%% Horizontal Lines
\\ \cline{1-1}\cline{3-3}\cline{5-5}
%%
%%  12:00
%% Mo
12:00 \fibabstand\fibabstand\fibabstand& &
% Di & Mi
	\textbf{Buch-Club}\fibnlx
	(Fachliteratur)\fibnl
	\hspace*{\fill}\textit{Hörsaal~AP} & &
% Do & Fr
	\textbf{Laborführungen}\fibnl
		\hspace*{\fill}
		\textit{Treffen vor der Fachschaft} &
%% Horizontal Lines
\\ \cline{1-1}\cline{3-3}\cline{5-6}
%%
%%  13:00
%% Mo
13:00 \fibabstand& &
% Di & Mi
	Mittagspause & &
% Do
	Mittagspause &
% Fr
	Mittagspause
%% Horizontal Lines
\\ \cline{1-3}\cline{5-6}
%%
%%  14:00
%%
14:00 \fibabstand &
% Mo
	\multirow{2}[2]{\fibprogrammcw}{\textbf{Infoveranstaltung~I}\fibnlx
		(Bachelor Physik/Geophysik, Zwei-Fach-Bachelor)
		\hspace*{\fill}
		\textit{Hörsaal~AP}} &
% Di & Mi
	\multirow{2}[2]{\fibprogrammcw}{\textbf{Campusspiel}\fibnl
		\hspace*{\fill}
		\textit{Treffen an der Freitreppe}} & & 
% Do
	\multirow{2}[2]{\fibprogrammcw}{\textbf{Ersti-Begrüßung des Rektorats}\fibnl
		\hspace*{\fill}
		\textit{Leonardo-Campus} 
	} &
	\multirow{2}[8]{\fibprogrammcw}{%
		\textbf{"Kaffeetrinken" mit den Professoren}\fibnl
		\hspace*{\fill}
		\textit{Foyer IG1}
	}
%% Horizontal Lines
\\ \cline{1-1}
%%
%%  15:00
%%
15:00 \fibabstand &
	& & & &
\\ \cline{1-2}\cline{5-5}
%%
%%  16:00
%%
16:00 \fibabstand &
% Mo & Di & Mi
	\multirow{4}[8]{\fibprogrammcw}{\textbf{Stadtspiel}\fibnl
		\hspace*{\fill}
		\textit{Treffen an der Freitreppe}\fibnlx\fibnlx\fibnlx
		anschließend Grillen vor der Fachschaft\fibnlx
		(bei passender Wetterlage)} & & & 
% Do & Fr
	\multirow{2}[2]{\fibprogrammcw}{%
		\textbf{Konstruktionswettbewerb}\fibnl
		\hspace*{\fill}
		\textit{Treffen im Foyer IG1}
	}&
%% Horizontal Lines
\\ \cline{1-1}
%%
%%  17:00
%%
17:00 \fibabstand & & & & &
%% Horizontal Lines
\\ \cline{1-1}\cline{3-3}\cline{5-6}
%%
%%  18:00
%% Mo
18:00 \fibabstand &	&
% Di & Mi
	\multirow{2}{\fibprogrammcw}{%
		\textbf{Grillen}\fibnlx
		(bei passender Wetterlage)
	} & &
% Do
	\multirow{3}[4]{\fibprogrammcw}{%
		\textbf{Spieleabend}\fibnlx
		Brettspiele, Kartenspiele, Pen~\&~Paper und mehr!\fibnl
		\hspace*{\fill}
		\textit{Seminarraum KP~104}
	} &
% Fr
	\textbf{Restegrillen}
\\ \cline{1-1}
%%
%%  19:00
%%
19:00 \fibabstand &	& & & &
%% Horizontal Lines
\\ \cline{1-1}
%%
%%  abends
%% Mo
abends\vspace{2\baselineskip} & &
% Di & Mi & Do & Fr
	\textbf{Kneipenabend}\fibnlx
		\hspace*{\fill}
		\textit{im "Buddenturm"} & & &
\\ \hline
\end{tabular}

\smallskip

% XXX Jedes Jahr Astroseminar-Termin aktualisieren!
\textbf{Du findest Astrophysik spannend?
	Zum Thema "Auf Entdeckungsreise im Kosmos" findet dieses Jahr wieder das jährliche Astroseminar am Fr.\ und Sa., 26.10. \& 27.10. (drei Wochen nach der Ersti-Woche) statt.
	Eintritt frei -- Anmeldung erforderlich!
	Mehr Infos: \url{https://www.uni-muenster.de/Astroseminar}}
\end{landscape}
