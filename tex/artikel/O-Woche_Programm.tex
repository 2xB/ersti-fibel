% XXX Jedes Jahr O-Wochen-Plan aktualisieren!
% Zeilenumbruch \\ als \fibnlx zwischenspeichern (in Tabellen wird \\ für eine
% neue Tabellenzeile umdefiniert)
\let\fibnlx=\\
% Befehl \fibnl ist ein Zeilenumbruch mit etwas Freiraum darunter
\DeclareDocumentCommand{\fibnl}{}{\fibnlx[\baselineskip]}
\DeclareDocumentCommand{\fibabstand}{}{\vspace{.5\baselineskip} }

% XXX Relevanz dieser Definition prüfen
% Nur für den Donnerstags-Einschub.
\newlength{\fibprogrammeinschub}
\setlength{\fibprogrammeinschub}{0.005\textheight}

% Länge \fibprogrammcw ist die Breite einer Spalte in der Programmtabelle
% (außer Spalte mit den Zeiten)
\newlength{\fibprogrammcw}
\setlength{\fibprogrammcw}{0.214\textheight - 0.25\fibprogrammeinschub}


\begin{landscape}
\section{Programm der Physik-Orientierungseinheit (O-Woche/Ersti-Woche)}
\renewcommand{\arraystretch}{1.8}
\footnotesize
\begin{tabular}{
	|
	>{\bfseries\hfill} % füge „\bfseries\hfill“ zu Beginn jeder Zelle dieser Spalte ein (also fett und rechtsbündig)
	p{0.08\textheight} % Textspalte (mehrzeilig) der Breite 0.08\textheight
	|
	*{3} % füge die folgende Definition 3x ein
	{
		p{\fibprogrammcw} % Textspalte (mehrzeilig) der Breite \fibprogrammcw
		|
	}
	% Donnerstags-Einschub.
	p{\fibprogrammeinschub} % Textspalte (mehrzeilig) der Breite \fibprogrammeinschub
	|
	*{1} % füge die folgende Definition 1x ein
	{
		p{\fibprogrammcw} % Textspalte (mehrzeilig) der Breite \fibprogrammcw
		|
	}
}
\hline
%%
%% TABLE HEADER
%%

Uhrzeit &
	\textbf{Montag, 30.09.} &
	\textbf{Dienstag, 01.10.} &
	\textbf{Mittwoch, 02.10.} &
	\multirow{10}{*}{
		\hspace*{-2mm}\rotatebox{-90}{\hspace*{.07cm} \textbf{Donnerstag} ab 15 Uhr\textbf{: Konstruktionswettbewerb}}
	}&
	\textbf{Freitag, 04.10.}
%% Horizontal Lines
\\ \cline{1-4} \cline{6-6}
%%
%%  10:00
%%
10:00\fibabstand\fibabstand\fibabstand &
% Mo
	\multirow{4}{\fibprogrammcw}[-2mm]{%
		\textbf{Einführungsveranstaltung}\fibnl
		\hspace*{\fill}
		\textit{Hörsaal~AP}\fibnl
		\textbf{Tutorien und Institutsführung}\fibnl
		\textbf{Mittagessen}
	} & 
% Di
	\multirow{2}[2]{\fibprogrammcw}[-3mm]{%
		\textbf{Infoveranstaltung~II}\fibnlx
		(Gremien, IVV~NWZ+ZIV, BAföG)\fibnl
		\hspace*{\fill}
		\textit{Hörsaal~AP}} &
% Mi & Do
	\multirow{2}[6]{\fibprogrammcw}[-3mm]{%
		\textbf{Ausweichtermin Infoveranstaltung~I}\fibnlx
		(nur für Zwei-Fach-Bachelor)\fibnl
		\hspace*{\fill}
		\textit{Hörsaal KP~404}} & &
% Fr
%% Horizontal Lines
\\ \cline{1-1} \cline{6-6}
%%
%%  11:00
%% Mo & Di & Mi & Do
11:00 \fibabstand & & & & &
% Fr
	\textbf{Vortrag der Polizei}\fibnl
		\hspace*{\fill}\textit{Hörsaal~1}
%% Horizontal Lines
\\ \cline{1-1}\cline{3-3}\cline{4-4}\cline{6-6}
%%
%%  12:00
%% Mo
12:00 \fibabstand\fibabstand\fibabstand& &
% Di
	\textbf{Buch-Club}\fibnlx
	(Fachliteratur)\fibnl
	\hspace*{\fill}\textit{Hörsaal~AP} &
% Mi
	\textbf{Laborführungen}\fibnl
		\hspace*{\fill}
		\textit{Treffen vor der Fachschaft} & & 
% Do & Fr
	\multirow{2}[15]{\fibprogrammcw}{\textbf{Infoveranstaltung~III}\fibnlx
		(jDPG \& andere studentische Gruppierungen)\fibnlx[0.5em]
		\textbf{Plenum \& Preisverleihung}\fibnl
		\hspace*{\fill}
		\textit{Hörsaal~1}}
%% Horizontal Lines
\\ \cline{1-4}
%%
%%  13:00
13:00 \fibabstand& 
%% Mo
\multirow{2}[2]{\fibprogrammcw}{\textbf{Infoveranstaltung~I}\fibnlx
	(Bachelor Physik/Geophysik, Zwei-Fach-Bachelor)
	\hspace*{\fill}
	\textit{Hörsaal~AP}} &
% Di
	Mittagspause &
% Mi & Do
	Mittagspause & &
% Fr
%% Horizontal Lines
\\ \cline{1-1}\cline{3-4} \cline{6-6}
%%
%%  14:00
%%
14:00 \fibabstand &
% Mo
	&
% Di
	\multirow{4}[8]{\fibprogrammcw}{\textbf{Stadtspiel}\fibnl
		\hspace*{\fill}
		\textit{Treffen an der Freitreppe}
	}&
% Mi & Do
	\multirow{2}[2]{\fibprogrammcw}{\textbf{Ersti-Begrüßung des Rektorats}\fibnl
		\textit{Leonardo-Campus} \fibnlx
		\textit{Treffen an der Freitreppe}
	} & & 
% Fr
	\textbf{Restegrillen}
%% Horizontal Lines
\\ \cline{1-2}
%%
%%  15:00
%%
15:00 \fibabstand &
% Mo
\multirow{2}[2]{\fibprogrammcw}{\textbf{Campusspiel}\fibnl
	\hspace*{\fill}
	\textit{Treffen an der Freitreppe}}	&
% Di & Mi & Do & Fr
 & & &
\\ \cline{1-1}\cline{4-4}
%%
%%  16:00
%%
16:00 \fibabstand &
% Mo & Di
	
	 & &
% Mi & Do & Fr
\multirow{2}[8]{\fibprogrammcw}{%
	\textbf{"Kaffeetrinken" mit den Professoren}\fibnl
	\hspace*{\fill}
	\textit{Foyer IG1}
} & &
%% Horizontal Lines
\\ \cline{1-1} \cline{6-6}
%%
%%  17:00
%%
17:00 \fibabstand & & & & &
%% Horizontal Lines
\\ \cline{1-1}\cline{3-3}\cline{6-6}
%%
%%  18:00
18:00 \fibabstand &
%% Mo
\multirow{2}{\fibprogrammcw}{%
	\textbf{Grillen}\fibnlx
	(bei passender Wetterlage)
} &
% Di & Mi
\textbf{Kneipenabend}\fibnlx
\hspace*{\fill}
% XXX Jedes Jahr Räumlichkeit aktualisieren!
\textit{in der "Cavete"} & &
% Do & Fr
\multicolumn{2}{p{\fibprogrammcw}|}{
\multirow{3}[4]{\fibprogrammcw}{%
	\textit{anschließend}\fibnlx
	\textbf{Spieleabend}\fibnlx
	Brettspiele, Kartenspiele, Pen~\&~Paper und mehr!\fibnlx
	\hspace*{\fill}
	\textit{Seminarraum KP~104}
}
}
%% Horizontal Lines
\\ \cline{1-1}
%%
%%  abends
%% Mo
abends\vspace{2\baselineskip} & &
% Di & Mi & Do & Fr
& & \multicolumn{2}{p{\fibprogrammcw}|}{}
\\ \hline
\end{tabular}

\smallskip

% XXX Jedes Jahr Astroseminar-Termin aktualisieren!
\textbf{Du findest Astrophysik spannend?
	Zum Thema "Dem Unsichtbaren auf der Spur" findet dieses Jahr wieder das jährliche \mbox{Astroseminar} am Fr.\ und Sa., 25.10. \& 26.10. (drei Wochen nach der Ersti-Woche) statt.
	Eintritt frei -- Anmeldung beachten!\\
	Mehr Infos: \url{https://www.uni-muenster.de/Astroseminar}}
\end{landscape}
