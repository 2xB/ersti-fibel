% Autor: Simon May
% Datum: 2014-08-19

% "\textasciitilde" = ~
\section[Fahrradfahren in Münster]{Crashkurs zum Vermeiden von Crashs\\\textasciitilde\textasciitilde\textasciitilde\ Fahrradfahren in Münster \textasciitilde\textasciitilde\textasciitilde}
\begin{multicols}{2}
% subsections in diesem Artikel kleiner (mit "\normalsize") darstellen,
% sonst ist zu wenig Platz
\addtokomafont{subsection}{\normalsize}
\textbf{Sicher habt ihr es auch schon gemerkt -- in Münster sind überall Fahrradfahrer. Anfangs sieht es so aus, als sei das ein riesiges Chaos. Ihr werdet sicherlich nicht alles richtig machen, wenn ihr euch aufs Rad schwingt. Wer rechnet schließlich damit, dass man immer auf der rechten Hälfte des roten Radweges fahren muss, wenn man nicht überholen will. Nach kurzer Zeit werdet ihr jedoch bestimmt alle Besonderheiten des Radfahrens in Münster verinnerlicht haben. Um euch das zu erleichtern, sind hier ein paar nützliche Lektionen:} 

\subsection*{Lektion 1: Ampeln}
An fast jeder Kreuzung gibt es auch Ampeln für Fahrradfahrer. Ihr dürft allerdings meist nur auf der rechten Straßenseite über die Kreuzung fahren, sonst müsst ihr das Rad schieben. Aufgepasst: Die Ampeln sind weder mit den Fußgänger- noch mit den Autoampeln synchronisiert. Also achtet immer auf die speziellen Fahrradampeln und auch darauf, ob ihr geradeaus oder rechts fahren wollt, denn auch das sind nicht immer die gleichen Ampeln.

Sollte es einmal keine Fahrradampel geben, so gelten die Ampeln für Autos, wenn die Ampel rechts vom Radweg ist und die für die Fußgänger, wenn die Ampel links vom Radweg ist. Das ist allerdings vielen Radfahrern nicht bekannt oder einfach egal. Davon solltet ihr euch nicht irritieren lassen.

\subsection*{Lektion 2: Fahrbahnmarkierungen}
Generell gilt: die roten Wege sind Fahrradwege. Doch gerade in der Nähe der Uni gibt es viele zusätzliche weiße Fahrbahnmarkierungen. Vor der Mensa~am~Ring dürft ihr zum Beispiel in beide Richtungen fahren, allerdings nur direkt davor. Über die Ampeln an den Kreuzungen vor der Mensa dürft ihr größtenteils in beide Richtungen fahren und braucht nicht zu schieben. Hier werdet ihr anfangs -- und auch später noch -- oft die Übersicht verlieren.

Grundsätzlich gilt: Immer dem roten Weg bzw. den weißen Markierungen folgen. Doch ab und an teilt sich ein roter Weg in zwei oder eine weiße Markierung hört plötzlich auf. Wenn es zwei Wege gibt, gilt: der linke ist für den Gegenverkehr, der rechte für euch. Wenn eine weiße Markierung aufhört, heißt es für euch: Absteigen, falls ihr auf der linken Straßenseite seid.

\subsection*{Lektion 3: Fahrradstraßen}
Einige Straßen sind spezielle Fahrradstraßen und auch als solche gekennzeichnet. Das bedeutet, dass ihr in den Straßen Vorfahrt habt. Aber Vorsicht: Nicht alle Autofahrer bemerken, dass sie in einer Fahrradstraße sind\dots

\subsection*{Lektion 4: Fahrradständer}
So viele verschiedene Prinzipien an Fahrradständern gibt es wohl in kaum einer Stadt. Grundsätzlich gilt hier: Ihr müsst das Vorderteil in den Ständer einhängen. Sei es, dass ihr das Rad zwischen den Speichen einhängt, oder den Lenker auf den Ständer legt. Auch wenn es anfangs umständlich scheint, das Fahrrad einzuhängen -- ihr werdet spätestens nach dem ersten Dominoeffekt, bei dem alle auf dem eigenen Ständer stehenden Fahrräder umstürzen, merken, dass es sich lohnt, die zwei Sekunden Zeit aufzuwenden.

\subsection*{Lektion 5: Regen}
Bevor der Winter beginnt, solltet ihr euch eine geeignete Regenjacke und eine Regenhose zulegen. Nichts ist unangenehmer, als in den Vorlesungen zu frieren, weil man keine Funktionskleidung hat. Passt auf, dass ihr die Kleidung nach den Vorlesungen wieder einpackt, wenn ihr sie zum Trocknen aufgehängt habt. Man merke: Das Aussehen ist unwichtig, solange die Kleidung funktioniert.

\subsection*{Lektion 6: Diebstahl}
In Münster werden sehr viele Fahrräder und auch Fahrradutensilien gestohlen. Deshalb beachtet, dass ihr ein sicheres Schloss benutzt. Fahrradkörbe, -taschen und -beleuchtung solltet ihr entweder sicher befestigen oder vom Fahrrad entfernen, bevor es unbeaufsichtigt stehen bleibt. Nutzt auch das Angebot der Polizei, das Fahrrad registrieren zu lassen. Hierzu gibt es immer wieder Angebote in der Innenstadt. Ihr könnt euch die Unterlagen aber auch direkt bei der Polizeistation am Friesenring holen.

\subsection*{Lektion 7: Fahrradpolizei}
Die Polizei legt viel Wert darauf, dass die Fahrradfahrer sich an die Verkehrsregeln halten. Unter anderem gibt es etwa zehn Fahrradpolizisten, die mit dem Fahrrad durch die Stadt fahren und euch anhalten, wenn ihr zum Beispiel auf der falschen Seite fahrt, eine rote Ampel überfahrt oder Musik hört.

Damit ihr etwas besser kalkulieren könnt, was ihr euch erlauben könnt (für euren Geldbeutel und eure Punkte in Flensburg) und was ihr besser lassen solltet, könnt ihr im Internet unter

% URL etwas einrücken und linksbündig setzen
% (URLs am besten immer linksbündig, rechtsbündig oder zentriert setzen,
% sonst meckert LaTeX, da es Zeilenumbrüche nur an "/" oder "." macht)
\begin{flushleft}
% etwas einrücken
\hangafter=0\hangindent=0.5cm
\url{http://www.adfc-sh.de/htdocs/jo-sh/index.php/component/content/article/16-oeffentlichkeitsarbeit/175-bussgeldkatalog-2013}
\end{flushleft}

oder bei den Aushängen vor dem Fachschaftsraum nachlesen. Seid ihr noch in der Probezeit, solltet ihr nicht über rote Ampeln fahren und beim Zebrastreifen anhalten, wenn Fußgänger kommen. Hierfür gibt es nämlich Punkte in Flensburg.

Und jetzt bleibt nur noch zu wünschen, dass ihr euch vom Fahrradfieber anstecken lasst ;-)

\fibelsig{Judith}
\end{multicols}
