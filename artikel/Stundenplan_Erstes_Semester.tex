% Autor: Simon May
% Datum: 2014-09-14

\section{Dein Stundenplan im 1.\ Semester}
\vspace{-0.15cm}
\subsubsection*{für 1-Fach-Bachelor-Physiker, 2-Fach-Bachelor-Physiker und Geophysiker}
\begin{minipage}{\textwidth}
% Keine Trennlinie vor Fußnoten in dieser minipage
\setfootnoterule{0cm}
% Die Länge "\temp" ist die Breite einer Spalte in der Tabelle
% (außer Spalte mit den Zeiten)
\setlength{\temp}{0.152\textwidth}
% Der Befehl "\fibnl" ist ein Zeilenumbruch
\let\fibnl\par

\centering
% Verlinkung von Fußnoten im PDF klappt nicht mit tabularx :-(
% geht mit tabular, tabular*
\begin{tabular}{| >{\footnotesize}p{0.1\textwidth} | *{5}{>{\footnotesize\centering\arraybackslash}p{\temp}|}}
\hline
	Uhrzeit &
	Montag &
	Dienstag &
	Mittwoch &
	Donnerstag &
	Freitag
\\ \hline
08:00--\fibnl 10:00 Uhr &
	\textbf{Physik~I Übung}\fibnl(diverse Seminarräume) &
	\textbf{Mathe für Physiker~I\footnote{Nicht für 2-Fach-Bachelor-Studenten.\label{stundenplan:mfp1}} Übung}\fibnl(diverse Seminarräume) &
	\textbf{Physik~I Vorlesung}\fibnl HS~1 &
	\textbf{Physik~I Übung}\fibnl(diverse Seminarräume) &
	Informatik~I\cref{stundenplan:informatik} Übung\fibnl(diverse Seminarräume)
\\ \hline
10:00--\fibnl 12:00 Uhr &
	\textbf{Mathe für Physiker I\cref{stundenplan:mfp1} Vorlesung}\fibnl HS~2 &
	\textbf{Physik~I Vorlesung}\fibnl HS~1 &
	&
	\textbf{Mathe für Physiker~I\cref{stundenplan:mfp1} Vorlesung}\fibnl HS~2 &
	\textbf{Physik~I Vorlesung}\fibnl HS~1
\\ \hline
12:00--\fibnl 13:00 Uhr &
	Chemie\footnote{Für Physik-1-Fach-Bachelor mit dem Modul Chemie als fachübergreifende Studie.
	
	\noindent Die Termine zu den Übungen werden in der Vorlesung verteilt.\label{stundenplan:chemie}} Vorlesung\fibnl C1 &
	Chemie\cref{stundenplan:chemie} Vorlesung\fibnl C1 &
	Chemie\cref{stundenplan:chemie} Vorlesung\fibnl C1 &
	Chemie\cref{stundenplan:chemie} Vorlesung\fibnl C1 &
\\ \hdashline
13:00--\fibnl 14:00 Uhr &
	& & & &
\\ \hline
14:00--\fibnl 16:00 Uhr &
	Informatik~I\footnote{Für Physik-1-Fach-Bachelor mit dem Modul Informatik als fachübergreifende Studie.
	
	\noindent Die Termine zu den Übungen werden in der Vorlesung verteilt.\label{stundenplan:informatik}} Vorlesung\fibnl M~1 &
	&
	Einführung in die Geophysik\footnote{Für Geophysik-1-Fach-Bachelor und Physik-1-Fach-Bachelor mit dem Modul Geophysik als fachübergreifende Studie.
	
	\noindent Die Termine zu den Übungen werden in der Vorlesung verteilt.\label{stundenplan:geophysik}} Vorlesung\fibnl HS~AP &
	Informatik~I\cref{stundenplan:informatik} Vorlesung\fibnl M~1 &
\\ \hline
16:00--\fibnl 18:00 Uhr &
	& & & &
\\ \hline
18:00--\fibnl 20:00 Uhr &
	&
	&
	Fachschafts-Sitzung\fibnl(Ihr seid alle willkommen!) &
	&
\\ \hline
\end{tabular}
\end{minipage}

\vspace{-0.2cm}
Des Weiteren können 1-Fach-Bachelor-Studenten, die weder Chemie, Geophysik, noch Informatik als fachübergreifende Studie wählen möchten, aus folgenden vorgefertigten Modulen als fachübgreifende Studien wählen:
\begin{itemize}[parsep=0.1cm]
\item Philosophie für Physiker
\item Theoretische Grundlagen der Psychologie
\item Einführung in die Betriebswirtschaftslehre
\item Einführung in die Volkswirtschaftslehre
\item Spanisch für Naturwissenschaftler
\item (Deutsch als Fremdsprache)
\end{itemize}
oder ein selbstzusammengestelltes fachübergreifendes Modul, bei dem der Zusammenhang zur Physik erkennbar ist, oder das zur Berufsfähigkeit dient, bestimmen. Dies ist aber nur nach Absprache mit dem Studiendekan möglich (aktuell: Prof.\ Dr.\ Gernot Münster, Institut für Theoretische Physik (TP), Raum 317). Die genauen Vorlesungs-/Übungszeiten der zusätzlichen fachübergreifenden Studien erfahrt ihr in der O-Woche von der Fachschaft Physik, direkt von der jeweiligen Fachschaft des zuständigen Fachbereiches, oder ihr schaut einfach im Vorlesungsverzeichnis (HIS LSF, \url{https://studium.uni-muenster.de/qisserver/rds?state=wtree&search=1}) der Uni nach.

\fibelsig{Simon}
