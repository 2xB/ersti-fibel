% Autor: Simon May
% Datum: 2014-09-17
\section{Ein Erstiwochenende fernab von Raum und Zeit}
\textbf{Hier erzählen euch nun zwei "alte" Erstsemster-Studentinnen ihre Erfahrungen von einem vergangenen Erstsemester-Wochenende in der spektakulären "Partyhochburg" in Münster-Roxel.\\
HINWEIS: Das Erstiwochenende findet dieses Semester vom 14.--16.11.\ statt. (Verbindliche) Anmeldung in der Fachschaft; wir freuen uns auf euch!}
\begin{multicols}{2}
Hallo liebe Erstis, wir berichten euch über unser schönes und sehr informatives Erstiwochenende.
\begin{tikzpicture}[remember picture, overlay]
\node[inner sep=0, xshift=-1.2cm, yshift=4cm] at (current page.center)
{\includegraphicscompressed[width=7cm]{res/erstiwe_gnom.png}};
\end{tikzpicture}

% XXX
%% wrapfigure verwendet "\columnsep" für den Abstand zwischen Text und Bild 
%\setlength{\columnsep}{0.1cm}
%\begin{wrapfigure}[15]{l}[1cm]{0cm}
%\includegraphics[width=6cm]{res/erstiwe_gnom.jpg}
%\end{wrapfigure}


\parshape=15
0cm 6.7cm
0cm 6.5cm
0cm 6cm
0cm 5.5cm
0cm 5.5cm
0cm 5.5cm
0cm 5.5cm
0cm 4.7cm
0cm 4.5cm
0cm 4.5cm
0cm 4.6cm
0cm 4.8cm
0cm 4.8cm
0cm 4.8cm
0cm \columnwidth
An einem Freitagnachmittag sind wir zusammen mit der Fachschaft gefahren. Nach unserer Ankunft begrüßte uns der Herbergsvater, und zeige uns die Unterkunft in Greven. Als wir mit dem Bettenbeziehen fertig waren, trafen wir uns zum Abendessen. Anschließend stand eine Infoveranstaltung auf dem Plan. Danach ließen wir den Abend mit einer lustigen Spielrunde ausklingen. Um 22~Uhr war Nachtruhe.

Gleich nach dem Aufstehen war eine Stadtführung durch Emsdetten angesetzt und das Nachmittagsprogramm sah interessante Vorträge über komplexe Themengebiete der Physik vor.

Sonntags nach dem Frühstück ging es dann fix zurück ins schöne Münster.

\begin{quote}
\textit{Es war wirklich ein sehr informatives Wochenende!!!}
\end{quote}

Okay, wer sich jetzt denkt: "Ohhh, auf so ein tolles, informatives Wochenende will ich auch fahren", den müssen wir an dieser Stelle enttäuschen.

Hier die wahre unzensierte Version des absolut geilen Erstiwochenendes!!!!

Am Freitag, den 28.11.2008 haben sich knapp 30~vorfreudige Erstis am Bahnhof getroffen, um gemeinsam in ein spannendes und lustiges Wochenendabenteuer zu starten. Bereits während der Fahrt zur Herberge wurde das eine oder andere Bier oder manch gewisser Feigling getrunken~;-)\\
Unser Ziel für dieses Wochenende war eine kleine Jungendherberge im umgebauten Kloster Hiltrup.

\parshape=16
0cm \columnwidth
0cm \columnwidth
0cm \columnwidth
0cm \columnwidth
0cm \columnwidth
0cm \columnwidth
0cm \columnwidth
0cm \columnwidth
0cm \columnwidth
0cm \columnwidth
0cm \columnwidth
0cm \columnwidth
2.2cm \dimexpr\columnwidth - 2.2cm
2.2cm \dimexpr\columnwidth - 2.2cm
2.2cm \dimexpr\columnwidth - 2.2cm
0cm \columnwidth
Nach einer netten Begrüßung von Seiten der Fachschaft haben wir unsere Zimmer bezogen, welche wir bunt gemischt belegen konnten, wie es uns gefiel. Jetzt, knapp 9~Monate nach dem Wochenende, stellt sich raus, dass die wild gemischten Zimmer keine falsche Entscheidung waren, denn wie man sieht, ist ja kein Baby entstanden\dots\ obwohl, wer weiß eigentlich, was aus Chantal geworden ist?? Egal, weiter im Text{\dots} Wer befürchtet, während des Wochenendes auf dem Trockenem zu sitzen, dem sei gesagt, dass wir vom Herbergsvater aus reichlich mit preisgünstigem Bier versorgt wurden. Aber nun wieder zum Abendverlauf. Nach der Zimmerbelegung trafen wir uns im Gemeinschaftsraum, um uns über diverse Spiele besser kennenzulernen, welche uns zum Teil schon aus der Kindheit bekannt waren.

Da die Jungendherberge von einem großen Wald umgeben ist, war der nächste Punkt des Abends ein Waldspiel, wo wir in kleinen Gruppen ohne Taschenlampe bewaffnet durch den wirklich dunklen Wald laufen, oder besser stolpern und viele spaßige Aufgaben bewältigen mussten, was gar nicht so leicht war mit steigendem Alkoholpegel :-P Als alle wieder heile aus dem dunklen Wald zurückgekehrt waren, wurden in gemütlicher Runde noch ein paar Bierchen getrunken.\\
Das merkte man dann auch am nächsten Morgen\dots\ wir sagen nur Katerstimmung!!

\begin{center}
\includegraphicscompressed[width=0.65\columnwidth]{res/erstiwe_weihnachtsmann.png}
\end{center}

Manche bekämpften ihn, indem sie sich nach dem Frühstück noch mal hinlegten, andere gingen ins Dorf. Dieser Spaziergang diente natürlich dem Zweck, sich in der kühlen Novemberluft ein klaren Kopf zu beschaffen, aber auch, um in den Supermärkten von Sprakel einzukaufen.

\begin{center}
\includegraphicscompressed[width=0.65\columnwidth]{res/erstiwe_gasmaske.jpng}
\end{center}

Nach dem Mittagessen, welches für eine Herberge echt lecker war, hatten wir jetzt wirklich zwei Infoveranstaltungen. Eine zum Thema Studienparlament und eine Diskussionsrunde über Studiengebühren. Diese waren aber wirklich nicht in die Länge gezogen und auch recht informativ. In der Zeit bis zum Abendessen konnten wir wieder frei machen, was wir wollten.

\includegraphicscompressed[width=\columnwidth]{res/erstiwe_kartenspiel.png}

Am Abend gab es dann das berüchtigte und wirklich sehr chaotische "Chaosspiel", welches seinem Namen mehr als Ehre gemacht hat. Da das Spiel sehr lange dauert, und man immer wieder an den Bierkästen vorbei laufen musste, verwundert es nicht, dass am Ende des Spiels jeder wieder gut dabei war.

Nach diesem Spiel wurden die Sieger der bisherigen Spiele gekürt und mit grandiosen Geschenken überhäuft! Der Abend hat noch viele weitere spannende Überraschungen für uns parat gehabt, doch wollen wir nicht zu viel erzähle\dots\ ein bisschen müsst ihr euch ja auch überraschen lassen~;-)

\includegraphics[width=\columnwidth]{res/erstiwe_strohhalme.jpg}

Nur, dass der Abend in einer lustigen und erinnerungswürdigen Party geendet hat, wollen wir an dieser Stelle verraten!

Am nächsten Morgen hieß es dann, die Spuren der letzten Tage zu beseitigen und die Herberge blitzeblank dem Herbergsvater wieder zu übergeben, um die Heimreise antreten zu können.

Wieder im Unialltag angekommen, waren alle, die auf diesem absolut geilen Wochenende dabei waren, traurig, dass es nur so kurz war und sehnten sich zurück in die Partyhochburg Münster-Roxel.

Wir können euch nur empfehlen, dieses Jahr auch bei dem Erstiwochenende mitzufahren, so wie wir vor einem Jahr. Man lernt viele interessante Leute aus dem eigenen Studiengang und Semester kennen, hat irre viel Spaß und noch ein Jahr später eine schöne Erinnerung.

\begin{center}
\large\Fontlukas{An dieser Stelle wollen wir der Fachschaft nochmal für das wirklich toll organisierte und lustige Wochenende im letzten Jahr danken!}
\end{center}

\fibelsig{Annika, Silke}
\end{multicols}

