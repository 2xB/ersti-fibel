% Autor: Simon May
% Datum: 2014-09-17
\section{Internet, NWZnet, ComputerLabs \& Co.}
\begin{multicols}{2}
\subsection*{How to get Internet -- ten years ago\dots}
Netzwerke sind inzwischen der wohl wichtigste Teil der IT-Welt und ein wesentlicher Bestandteil des täglichen Handwerkszeugs im wissenschaftlichen Umfeld. Das bekannteste Netzwerk neben dem Telefon- und Stromnetz ist das Internet. Für Studierende an der Universität~Münster gibt es mehrere Möglichkeiten, Zugriff auf das Internet und die verschiedenen Dienste zu bekommen. Ein Weg, der allen Studierenden offen steht, ist der Zugang über das Projekt~DaWIN. DaWIN, nicht zu verwechseln mit dem britischen Naturforscher Darwin, ist die Abkürzung für "Datenkommunikation für Studierende im WIssenschaftsNetz". Es handelt sich dabei um eine studentische Initiative an der WWU, die Studierenden aller Fachbereiche die Vorteile der weltweiten elektronischen Kommunikation zur Verfügung stellt.

Gerade für Physikstudenten ist es unabdingbar, sich mit dieser Art der Kommunikation vertraut zu machen, da ohne den Rechner und die lokalen wie weltweiten Netzwerke eine Arbeit oder Forschung im naturwissenschaftlichen Bereich in Gegenwart und Zukunft undenkbar ist.

\subsubsection*{Account, was ist das überhaupt?}
Um Zugang zum Rechnernetz der Uni und damit in die weite Welt zu bekommen, musste man früher zunächst einen Nutzerantrag des Zentrums~für~Informationsverarbeitung~(ZIV) ausfüllen.
%%%% 2014-08-10 Simon: Ich hab das Ganze mal deutlich gekürzt, weil heute irrelevant
%Das Antragsformular dazu gibt es am Dispatch des ZIV in der Einsteinstraße 60 (wenn man reinkommt links, der Schalter mit Glasscheibe und Klingel). Beim Ausfüllen ist zu beachten, unter Punkt 4. des Antrages, neben den Diensten „für Studierende/Schüler(innen) an Hochschulen in Münster (u0dawin usw.)“, auch den „Zugang für Studierende zu den Rechnersystemen der Fachbereiche/Fächer Physik (p0stud)“ anzukreuzen. Ist der DaWINZugang für Studierende anderer Fachbereich ausreichend, so benötigt man in den Naturwissenschaften einen speziellen Account, der auch Zugriff auf die Bereichsrechner und CIPPools gewährt. Dafür wurde für die drei naturwissenschaftlichen Fachbereiche je eine Studierendengruppe angelegt, die man an oben bezeichneter Stelle als Projekt angeben muss. Für die Physik ist dies p0stud.
%
%Nach ein paar Tagen (je nach Andrang) kann man das bearbeitete Formular am Dispatch wieder abholen. Auf der Bestätigung über den Netzzugang ist eine Nutzerkennung (der Account) angegeben, mit der man sich dem Rechner gegenüber identifiziert und ein Passwort, das man bei der ersten Benutzung ändern muss. Der Account stellt gleichzeitig die E-Mailadresse dar, erweitert um das obligatorische @uni-muenster.de. Unter dieser Adresse (Die darf man weiter geben, das Passwort nie!!) ist man nun weltweit zu erreichen.
Heute erhaltet ihr bereits mit eurer Einschreibung in die Uni eine Nutzerkennung der Form \texttt{v\_nnnn\#\#} mit zugehörigem Passwort, die aus eurem Vor- und Nachnamen sowie zwei Zahlen gebildet wird -- für "Max Mustermann" wäre eine mögliche Kennung "\texttt{m\_must08}". Zusätzlich erhaltet ihr eine Uni-E-Mail-Adresse -- hängt für diese einfach \texttt{@uni-muenster.de} oder \texttt{@wwu.de} (funktioniert beides) an eure Nutzerkennung an.

Nutzerkennungen der Uni werden zudem verschiedenen Nutzergruppen zugewiesen; für Studierende der Physik sind das zunächst \texttt{u0dawin} (Studierende) und \texttt{p0stud} (Angehörige der Physik). So werden bspw. Angebote auf Studierende bestimmter Fachbereiche beschränkt. Allgemein müsst ihr euch aber nicht mit Nutzergruppen auskennen -- das ist eher Detailwissen und passiert im Wesentlichen im Hintergrund.

\includegraphics[width=\columnwidth]{res/comics/hacker.png}\\
{\scriptsize\url{http://www.friedenspaedagogik.de}}

\subsubsection*{Was kann man denn nun alles machen?}
Der wichtigste Dienst ist sicherlich die E-Mail (Electronic-Mail); besonders praktisch, wenn man Bekannte und Freunde an anderen Unis oder in anderen Städten hat, die einen E-Mail Zugang haben. Man kann so ständig mit ihnen in Kontakt bleiben, ohne horrende Porto- oder Telefongebühren berappen zu müssen. Aber auch uni-intern kann man sich mit Leuten aus anderer Fachbereiche häufig schneller über das Netz verständigen. Die Fachschaft~Physik z.B. ist unter der Adresse \email{fsphys@uni-muenster.de} zu erreichen.

%%%% 2014-08-10 Simon: Heute ebenfalls kaum noch interessant
%Auch sehr nützlich sind die sogenannten NetNews, eine Art Zeitung im Internet, mit derzeit bei uns über 10.000 Rubriken, von denen man sich die auswählt, die man lesen möchte. Das werden wohl im wesentlichen die lokalen Gruppen „wwu.blablabla“ und „muenster.wasweißich“ oder bekannte überregionale, wie etwa die des Maus-„Netzes“ sein.
%
%Das Phänomenale daran ist jedoch, da man sofort auf jeden Artikel (Posting) antworten kann (followup) oder selbst Artikel „posten“ kann.
%
%Außerdem hat man Zugriff auf das SISIS-JOPAC, das integrierte Katalog- und Ausleihsystem der ULB, man braucht also nicht jedes Mal zur ULB zu tigern, um seine Bücher zu verlängern.
%
%Gerade für diejenigen, die sich mit dem Medium Computer bisher noch gar nicht auseinander gesetzt haben, wird in der zweiten oder dritten Vorlesungswoche eine „Rechnereinführung“ speziell für Erstsemester bzw. Neulinge am Fachbereich Physik angeboten werden. Der genaue Termin wird noch durch Aushang bzw. Ansage in der Vorlesung bekanntgegeben.

\subsection*{How to get Internet -- today}
Inzwischen wird wie erwähnt für Studierende an der WWU der Zugriff auf das Internet und eine E-Mail-Adresse automatisch bei der Einschreibung eingerichtet. Ihr seid verpflichtet, regelmäßig die Mails zu lesen, um Informationen von der Universität zu erhalten. Zum Beispiel kommt die Anfrage auf Rückmeldung zum nächsten Semester nur noch per E-Mail, d.h.\ ihr werdet aufgefordert, die Gebühren für das kommende Semester zu überweisen, um nicht exmatrikuliert zu werden. Auch Informationen zu Fristen bei der Prüfungsanmeldung im QISPOS oder (natürlich das wichtigste ;-) ) Mails von uns, der Fachschaft, bekommt ihr natürlich nur an eurer Uni-Postfach. Mails an euer Uni-Postfach könnt ihr auch problemlos weiterleiten lassen, falls euch die Nutzung eines Mail-Programms oder das manuelle Nachsehen zu aufwendig ist.

\includegraphics[width=\columnwidth]{res/comics/computersuechtig.jpg}

\subsubsection*{Portale}
Im Portal myWWU (\url{https://www.uni-muenster.de/mywwu}) sind die wichtigsten Dienste der Uni zusammengefasst. Dazu gehören das E-Mail-Postfach, das Vorlesungsverzeichnis (HIS~LSF), ein Kalender (der euch auch erinnert, wann ihr Bücher zurückgeben müsst), \dots

Auch sehr nützlich ist der Zugriff auf den OPAC, ein integriertes Katalog- und Ausleihsystem der ULB, und disco (\url{https://disco.uni-muenster.de}), das neue Suchsystem der ULB. Hier sind der OPAC und viele weitere Verzeichnisse integriert, sodass ihr in vielen Millionen Dokumenten suchen könnt. Ihr braucht also nicht jedes Mal zur ULB zu laufen, um Bücher zu verlängern. Wesentlich wichtiger sind jedoch Buch- und Literaturrecherchen, die ihr schnell und effektiv per Netz an den verschiedensten Stellen machen könnt. Daneben gibt es über die ULB und den Fachbereich Physik auch einen kostenlosen Zugang zu allen wesentlichen wissenschaftlichen Zeitschriften und zu vielen Bücher z.B.\ von Springer.

\subsubsection*{Spielregeln}
Neben reiner Textinformation habt ihr natürlich einen ungefilterten Zugang zum Internet und damit Zugriff auf alle anderen Angebote. An dieser Stelle eine Warnung, damit es keine bösen Überraschungen gibt: Auch wenn der Zugang zum Internet über das Uninetz recht schnell ist und es viele Mitglieder an der Uni gibt, so können Urheberrechtsverstöße und andere illegale Aktivitäten zu euch zurückverfolgt werden. Auch der Versand von Spam und Viren führt z.B.\ zu einer Sperrung eines Netzzuganges. Auch die Sanktionen bei Zuwiderhandlung sind in der Benutzungsordnung geregelt. Dies soll euch nicht abschrecken, dennoch solltet ihr die Spielregeln kennen.

\subsubsection*{ComputerLabs}
Es gibt eine Vielzahl von Rechnern an der Uni, doch nicht auf allen habt ihr Zugang. Der Fachbereich Physik hat, auf mehrere Gebäude verteilt, "ComputerLabs" eingerichtet, für die ihr den eingangs erwähnten speziellen Zugang benötigt. Mit diesem Zugang könnt ihr übrigens auch die Rechner in den Labs der Biologie und Chemie benutzen und umgekehrt die Studierenden dieser Fachbereiche auch "unsere" Rechner. Dass ihr überall dieselbe Arbeitsumgebung, euer Netzlaufwerk (Laufwerk \texttt{I:} mit \SI{10}{GB} Speicherplatz) und die gleichen Programme vorfindet, dafür ist gesorgt. Außerdem gibt es noch allgemein zugängliche ComputerLabs wie die im ZIV~(Einsteinstraße~60). Diese können von allen Angehörigen der Uni verwendet werden, allerdings stehen euch hier nicht dieselbe Software-Auswahl und Arbeitsumgebung zur Verfügung wie bei den Rechnern des naturwissenschaftlichen Zentrums. Standardmäßig ist beispielsweise nur euer ZIV-Netzlaufwerk (Laufwerk \texttt{U:} mit \SI{1}{GB} Speicherplatz) und nicht das I-Laufwerk eingebunden.

Die ComputerLabs der Physik findet ihr an folgenden Orten:
\begin{description}
\item[Angewandte Physik:] 8~Windows-PCs.
\item[Institut~für~Kernphysik,] 2.~Stock: 12~Windows-PCs, Scanner, s/w-Laserdrucker.
\item[Institut~für~Theoretische~Physik,] 4.~Stock: 9 Windows-PCs.
\item[Institutsgruppe~1~(IG1):]~\par
	\begin{itemize}[leftmargin=0.2cm]
	\item StudiBib, Erdgeschoss, Raum~13: Zwei Windows-PCs, Scanner, Farbdrucker.
	\item Institut für Technik und ihre Didaktik (Raum~220): 10~Windows-PCs, Scanner, s/w-Laserdrucker.
	\item Physikalisches~Institut, 5.~Stock, Räume~504 und 520: insgesamt 9~Windows-PCs, Dia-Scanner, A3-Flachbettscanner, s/w- und Farblaserdrucker.
	\item Materialphysik, Raum~613a: 3~Windows-PCs, s/w-Laserdrucker.
	\item Institut~für~Festkörpertheorie, Raum~745 und 747: insgesamt 17~Windows-PCs, Scanner, s/w-Laserdrucker.
	\end{itemize}
\item[Institut~für~Geophysik,]~\\Corrensstraße, Raum~333.
\item[Seminar für Didaktik des Sachunterrichts]~\\(DDSU) im Leonardo-Campus~11, Raum~104: 10~Windows-PCs, s/w-Laserdrucker.
\end{description}

Die jeweiligen Ansprechpartner bei Fragen, Problemen und auftretenden Fehlern sind im jeweiligen ComputerLab bekanntgegeben. Auf all diesen Computern ist ein sehr umfangreiches Software-Angebot installiert, sodass ihr dort direkt arbeiten könnt. Eine Übersicht gibt es auf der Internetseite der IVV~NWZ~\cref{internet:ivvnwz}.

Ein kurzer Hinweis zum Begriff "IVV": Neben dem ZIV, das für die zentrale Bereitstellung von IT an der Uni zuständig ist, gibt es an der Uni zehn  sogenannte dezentrale "Informations-Verarbeitungs-Versorgungseinheiten" (IVV), die für bestimmte Bereiche zuständig sind. Für die Fachbereiche Biologie, Chemie und Physik ist das die IVV~NWZ (IVV~4).

\subsubsection*{Weitere Angebote der\\IVV~NWZ und des ZIV}
Sowohl die IVV~NWZ als auch das ZIV betreiben Terminalserver, auf die ihr euch von Zuhause einwählen könnt. Damit habt ihr auch von Zuhause aus Zugriff auf die meiste Software und könnt damit arbeiten. Das Stichwort "Cloud" fällt meist irgendwann in diesem Zusammenhang und sollte heutzutage jedem bekannt sein. Etwas Ähnliches bieten IVV und ZIV schon seit vielen Jahren allen Studierenden an. Bei der IVV gibt es 10~GB Speicherplatz (Laufwerk~I) und beim ZIV 1~GB (Laufwerk U). Dieser Speicherplatz kann auch ähnlich wie eine Festplatte angebunden werden (WebDAV).

\includegraphics[width=\columnwidth]{res/comics/xkcd_computer_problems.png}

Auch kostengünstige Druckmöglichkeiten (A4 bis A0 und auch in Farbe) werden angeboten; für die Nutzung ist eine kostenlose Anmeldung bei "Print~\&~Pay" (in MeinZIV, Link im Abschnitt~"DPÜ") erforderlich. Weitere Informationen hierzu und vielen weiteren Angeboten sowie Anleitungen gibt es bei der IVV~NWZ~\cref{internet:ivvnwz} und beim ZIV~\cref{internet:ziv}, sowie im Info-Block~II der Ersti-Woche und bei uns~\cref{internet:fsphys_software}.

\subsubsection*{Software für Zuhause}
Für Studierende gibt es beim ZIV zwei interessante Softwarepakete, die insbesondere auch für den privaten Einsatz auf den eigenen Computern vorgesehen sind. Interessant ist vor allem die Anti-Virus-Software (Sophos), welche von Uni für alle bezahlt wird. Zusätzlich können viele weitere Programme, die das ZIV betreut, auch auf dem eigenen Rechner installiert und unter einigen Bedingungen genutzt werden~\cref{internet:ziv_software}.

Die IVV~NWZ ermöglicht zudem allen zugehörigen Studierenden den Zugriff auf das DreamSpark-Programm von Microsoft. Damit habt ihr Zugriff auf fast die gesamte Software-Palette von Microsoft, nur einige Office-Programme (Word, Excel, PowerPoint) dürfen nicht angeboten werden. Diese Software dürft ihr ausdrücklich auch auf privaten Rechner installieren. Weitere Programme wie die Mathematik-Software "Mathematica" (vielen vermutliche bereits durch die Webseite WolframAlpha bekannt) können ebenfalls unter verschiedenen Bedingungen bezogen werden -- Mathematica kann beispielsweise nur im Uni-Netzwerk (d.h.\ zum Beispiel über eine VPN-Verbindung) genutzt werden. Weitere Infos findet ihr wieder unter \cref{internet:ivvnwz}, \cref{internet:ziv} und \cref{internet:fsphys_software}.

\subsection*{Links}
\begin{flushleft}
\begin{fibelurl}
\url{https://www.uni-muenster.de/IVVNWZ}
\label{internet:ivvnwz}
\end{fibelurl}
\begin{fibelurl}
\url{https://www.uni-muenster.de/ZIV}
\label{internet:ziv}
\end{fibelurl}
\begin{fibelurl}
\url{https://www.uni-muenster.de/Physik.FSPHYS/service/software.html}
\label{internet:fsphys_software}
\end{fibelurl}
\begin{fibelurl}
\url{https://www.uni-muenster.de/ZIV/Software/Uebersicht.html}
\label{internet:ziv_software}
\end{fibelurl}
\end{flushleft}

\fibelsig{Judith, Simon}
\end{multicols}

