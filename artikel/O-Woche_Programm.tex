% Autor: Simon May
% Datum: 2014-09-09

% Zeilenumbruch "\\" als "\fibnlx" zwischenspeichern (in Tabellen wird
% "\\" für eine neue Tabellenzeile umdefiniert)
\let\fibnlx\\
% Befehl "\fibnl" ist ein Zeilenumbruch mit etwas Freiraum darunter
\DeclareDocumentCommand{\fibnl}{}{\fibnlx[\baselineskip]}

% Länge "\fibprogrammcw" ist die Breite einer Spalte in der Programmtabelle
% (außer Spalte mit den Zeiten)
\newlength{\fibprogrammcw}
\setlength{\fibprogrammcw}{0.214\textheight}
% Länge "\fibeltimeskip" ist der Freiraum in Zeilen, die etwas mehr Platz
% brauchen
\newlength{\fibeltimeskip}
\setlength{\fibeltimeskip}{3\baselineskip}

\begin{sideways}
\begin{minipage}{\textheight}
\section{Programm der Physik-Orientierungseinheit (O-Woche)}

\renewcommand{\arraystretch}{1.8}
\begin{tabular}{| >{\footnotesize\bfseries\hfill}p{0.06\textheight} | *{4}{>{\footnotesize}p{\fibprogrammcw} |}}
\hline
Uhrzeit &
	\textbf{Montag, 06.10.} &
	\textbf{Dienstag, 07.10.} &
	\textbf{Mittwoch, 08.10.} &
	\textbf{Donnerstag, 09.10.}
\\ \hline
10:00\vspace{\fibeltimeskip} &
	\multirow{2}[6]{\fibprogrammcw}{\textbf{Einführungsveranstaltung}\fibnl\hspace*{\fill}\textit{Hörsaal~AP}\fibnl} &
	\multirow{2}{\fibprogrammcw}{\textbf{Infoveranstaltung~II}\fibnlx(Gremien, NWZnet+ZIV, BAföG)\fibnl\hspace*{\fill}\textit{Hörsaal~AP}} &
	\multirow{2}[6]{\fibprogrammcw}{\textbf{Ausweichtermin Infoveranstaltung~I}\fibnlx(nur für Zwei-Fach-Bachelor)\fibnl\hspace*{\fill}\textit{Hörsaal KP~404}} &
\\ \cline{1-1}
11:00 & & & &
\\ \hline
12:00\vspace{\fibeltimeskip} &
	\multirow{2}[6]{\fibprogrammcw}{\textbf{Tutorien und Institutsführung\fibnl\fibnl Mittagessen}} &
	\multirow{2}{\fibprogrammcw}{Mittagspause} &
	\textbf{Laborführung}\fibnl\hspace*{\fill}\textit{Treffen vor der Fachschaft} &
	\multirow{2}{\fibprogrammcw}{\textbf{Infoveranstaltung~III}\fibnlx(jDPG, KSHG, \dots)\fibnl\textbf{Plenum \& Preisverleihung}\hspace*{\fill}\textit{HS~1}}
\\ \cline{1-1}\cline{4-4}
13:00 & & & Mittagspause & \\ \hline
14:00\vspace{\fibeltimeskip} &
	\multirow{2}[6]{\fibprogrammcw}{\textbf{Infoveranstaltung~I}\fibnlx(Bachelor Physik/Geophysik, Zwei-Fach-Bachelor)\fibnl\hspace*{\fill}\textit{Hörsaal~AP}} &
	\textbf{Buch-Club}\fibnlx(Fachliteratur)\hspace*{\fill}\textit{Hörsaal~AP} &
	\multirow{4}[30]{\fibprogrammcw}{\textbf{Stadtspiel}\fibnl\hspace*{\fill}\textit{Treffen an der Freitreppe}\fibnl\fibnl\fibnl\fibnl\fibnl anschließend Grillen vor der Fachschaft\fibnlx(bei passender Wetterlage)} &
	\multirow{2}{\fibprogrammcw}{\textbf{"Kaffeetrinken" mit Professoren}\fibnl\hspace*{\fill}\textit{Foyer IG~1}}
\\ \cline{1-1}\cline{3-3}
15:00 &
	&
	\textbf{Vortrag der Polizei}\fibnlx inkl. Fahrradregistrierung\hspace*{\fill}\textit{HS~AP} &
	&
\\ \cline{1-3}\cline{5-5}
16:00 &
	\multirow{2}{\fibprogrammcw}{\textbf{Physikspiel}\fibnl\hspace*{\fill}\textit{Treffen an der Freitreppe}} &
	\multirow{2}{\fibprogrammcw}{\textbf{Konstruktionswettbewerb}\fibnl\hspace*{\fill}\textit{Treffen im Foyer IG~1}} &
	&
\\ \cline{1-1}
17:00 & & & &
\\ \cline{1-3}
18:00\vspace{\fibeltimeskip} &
	\multirow{2}{\fibprogrammcw}{\textbf{Grillen}\fibnlx(bei passender Wetterlage)} &
	\textbf{ggf. DVD gucken}\fibnl\hspace*{\fill}HS~AP &
	&
\\ \cline{1-1}\cline{3-3}
abends &
	&
	\textbf{Kneipenabend 20:00}\fibnl\hspace*{\fill}\textit{Treffen vor der "Cavete"} &
	&
\\ \hline
\end{tabular}
\end{minipage}
\end{sideways}
