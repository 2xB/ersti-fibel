% Autor: Simon May
% Datum: 2015-07-31

% Zeilenumbruch "\\" als "\fibnlx" zwischenspeichern (in Tabellen wird
% "\\" für eine neue Tabellenzeile umdefiniert)
\let\fibnlx\\
% Befehl "\fibnl" ist ein Zeilenumbruch mit etwas Freiraum darunter
\DeclareDocumentCommand{\fibnl}{}{\fibnlx[\baselineskip]}

% Länge "\fibprogrammcw" ist die Breite einer Spalte in der Programmtabelle
% (außer Spalte mit den Zeiten)
\newlength{\fibprogrammcw}
\setlength{\fibprogrammcw}{0.214\textheight}
% Länge "\fibeltimeskip" ist der Freiraum in Zeilen, die etwas mehr Platz
% brauchen
\newlength{\fibeltimeskip}
\setlength{\fibeltimeskip}{3\baselineskip}

\begin{sideways}
\begin{minipage}{\textheight}
\section{Programm der Physik-Orientierungseinheit (O-Woche)}
\vspace{-0.7em}
\renewcommand{\arraystretch}{1.8}
\footnotesize{\begin{tabular}{| >{\footnotesize\bfseries\hfill}p{0.06\textheight} | *{4}{>{\footnotesize}p{\fibprogrammcw} |}}
\hline
Uhrzeit &
	\textbf{Montag, 12.10.} &
	\textbf{Dienstag, 13.10.} &
	\textbf{Mittwoch, 14.10.} &
	\textbf{Donnerstag, 15.10.}
\\ \hline
10:00\vspace{\fibeltimeskip} &
	\multirow{4}[25]{\fibprogrammcw}{\textbf{Einführungsveranstaltung}\fibnlx~\fibnlx
		\hspace*{\fill}\textit{Hörsaal~AP}\fibnl~\fibnlx
		\textbf{Tutorien und Institutsführung}\fibnl~\fibnl~\fibnl
		\textbf{Mittagessen}}\fibnl &
	\multirow{2}[8]{\fibprogrammcw}{\textbf{Infoveranstaltung~II}\fibnlx
		(Gremien, IVV~NWZ+ZIV, BAföG)\fibnl
		\hspace*{\fill}\textit{Hörsaal~AP}} &
	\multirow{2}[12]{\fibprogrammcw}{\textbf{Ausweichtermin Infoveranstaltung~I}\fibnlx
		(nur für Zwei-Fach-Bachelor)\fibnl
		\hspace*{\fill}\textit{Hörsaal KP~404}} &
\\ \cline{1-1}
11:00 & & & &
\\ \cline{1-1}\cline{3-5}
12:00\vspace{\fibeltimeskip} & &
	Mittagspause &
	\textbf{Laborführungen}\fibnl
		\hspace*{\fill}\textit{Treffen vor der Fachschaft} &
	\multirow{2}[16]{\fibprogrammcw}{\textbf{Infoveranstaltung~III}\fibnlx
		(jDPG, KSHG, \dots)\fibnl
		\textbf{Plenum \& Preisverleihung}\fibnl
		\hspace*{\fill}\textit{Hörsaal~1}}
\\ \cline{1-1}\cline{3-4}
13:00 & &
	 \textbf{Physik-Show\fibnlx
		 mit den Physikanten}\fibnl
		\hspace*{\fill}\textit{Hörsaal~AP} &
	 Mittagspause & \\ \hline
14:00\vspace{\fibeltimeskip} &
	\multirow{2}[12]{\fibprogrammcw}{\textbf{Infoveranstaltung~I}\fibnlx
		(Bachelor Physik/Geophysik, Zwei-Fach-Bachelor)\fibnl
		\hspace*{\fill}\textit{Hörsaal~AP}} &
	\textbf{Buch-Club}\fibnlx(Fachliteratur)\hspace*{\fill}\textit{Hörsaal~AP} &
	\multirow{4}[38]{\fibprogrammcw}{\textbf{Stadtspiel}\fibnl
		\hspace*{\fill}\textit{Treffen an der Freitreppe}\fibnl~\fibnl~\fibnl~\fibnl~\fibnl
		anschließend Grillen vor der Fachschaft\fibnlx
		(bei passender Wetterlage)} &
	\multirow{2}[8]{\fibprogrammcw}{\textbf{"Kaffeetrinken" mit Professoren}\fibnl
		\hspace*{\fill}\textit{Foyer IG~1}}
\\ \cline{1-1}\cline{3-3}
15:00 &
	&
	\textbf{Vortrag der Polizei}\fibnlx
		inkl.\ Fahrradregistrierung\hspace*{\fill}\textit{HS~AP} &
	&
\\ \cline{1-3}\cline{5-5}
16:00 &
	\multirow{2}{\fibprogrammcw}{\textbf{Physikspiel}\fibnl
		\hspace*{\fill}\textit{Treffen an der Freitreppe}} &
	\multirow{2}{\fibprogrammcw}{\textbf{Konstruktionswettbewerb}\fibnl
		\hspace*{\fill}\textit{Treffen im Foyer IG~1}} &
	&
\\ \cline{1-1}
17:00 & & & &
\\ \cline{1-3}
18:00\vspace{\fibeltimeskip} &
	\multirow{2}{\fibprogrammcw}{\textbf{Grillen}\fibnlx
		(bei passender Wetterlage)} &
	\textbf{Vorabendprogramm\fibnlx
		(Pizza Essen, "Vorglühen", \dots)}\fibnl
		\hspace*{\fill}\textit{Hörsaal~AP} &
	&
\\ \cline{1-1}\cline{3-3}
abends &
	&
	\textbf{Kneipenbingo 20:00}\fibnl
		\hspace*{\fill}\textit{Treffen vor der Cavete} &
	&
\\ \hline
\end{tabular}}

\smallskip

\textbf{Du findest Astrophysik spannend? Zu dem Thema "Unendliche Weiten, unendlich viel zu entdecken" findet in diesem Jahr das jährliche Astroseminar am Freitag und Samstag dieser O-Woche statt. Eintritt frei -- Anmeldung erforderlich! Mehr Informationen: \url{https://www.uni-muenster.de/Astroseminar}}
\end{minipage}
\end{sideways}
