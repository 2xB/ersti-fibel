% Autor: Simon May
% Datum: 2014-08-12

% Befehl "\fibelabk" fügt eine Abkürzug/einen Eintrag mit Erklärung ein
% 	Parameter #1: Abkürzung
% 	Parameter #2: Erklärung
\newcommand{\fibelabk}[2]{\textbf{#1} #2}

\section{Aküverz und Lexikon}
\begin{multicols}{2}
\textbf{Jeder kennt das Gefühl, wenn man mit scheinbar trivialen Abkürzungen überfordert ist und man sich selbst nur noch fragt, ob es nun peinlich wäre, nach dem Sinn zu fragen, oder ob man doch lieber schweigen und nicken sollte. Die erste Regel in solchen Situationen lautet "Don't panic!". Hier die wichtigsten Abkürzungen für dein Studium.}

\smallskip

% Abstand zwischen Paragraphen für den Rest des Artikels ändern
\setlength{\parskip}{0.03cm}

\fibelabk{42}{Antwort auf die große Frage nach dem Leben, dem Universum und Allem (siehe "Per Anhalter durch die Galaxis")}

\fibelabk{AG}{Arbeitsgemeinschaft, Arbeitsgruppe}

\fibelabk{AK}{Arbeitskreis}

\fibelabk{Aküverz}{Abkürzungsverzeichnis}

\fibelabk{AP}{Angewandte Physik}

\fibelabk{ASiUM}{Arbeitskreis Sicherheitspolitik an der Universität Münster, hochschulpolitische Gruppierung}

\fibelabk{AStA}{Allgemeiner~Studierendenausschuss; Interessensvertretung der Studierenden der Universität}

\fibelabk{Aus Symmetriegründen}{Abkürzung für "Diesen Beweis zu führen habe ich momentan weder Zeit noch Lust noch die Fähigkeit. Zudem würden Sie ihn ohnehin nicht verstehen. Man kann ihn aber in der einschlägigen Literatur nachschlagen."}

\fibelabk{BAföG}{Bundes-Ausbildungsförderungs-Gesetz}

\fibelabk{BaMa}{Bachelor-/Master-Studiengang}

\fibelabk{TBBT}{The Big Bang Theory}

\fibelabk{Burschenschaft}{siehe Verbindungen}

\fibelabk{C\textsubscript{2}H\textsubscript{5}OH}{Ethanol, beliebtes Genussmittel}

\fibelabk{CampusGrün~Münster}{hochschulpolitische Gruppierung}

\fibelabk{Computer}{vollkommen nutzloses Gerät, welches zur Vernichtung von Zeit entwickelt wurde; setzt immer dann aus, wenn der Artikel dringend weg muss}

\fibelabk{c.t.}{cum tempore; das akademische Viertelstündchen, d.h.\ Veranstaltungen fangen eine Viertelstunde später an}

\fibelabk{Dekan}{vertritt den Fachbereich; er wird vom Fachbereichsrat~(FBR) gewählt}

\fibelabk{DIL}{Demokratische Internationale Liste Münster, hochschulpolitische Gruppierung}

\fibelabk{DPG}{Deutsche Physikalische Gesellschaft; auch jDPG: junge Deutsche Physikalische Gesellschaft}

\fibelabk{Eva}{Evaluation der Lehre, siehe auch VU}

\fibelabk{FBR}{Fachbereichsrat}

\fibelabk{FK}{Fachschaftenkonferenz}

\fibelabk{F-Praktika}{Fortgeschrittenen-Praktika im Hauptstudium, auch: "Experimentelle Übungen für Fortgeschrittene"}

\fibelabk{FS}{Fachschaft; eigentlich alle Studierenden des Fachbereiches Physik; normalerweise versteht man unter "Fachschaft" den Fachschaftsrat (FSR) bzw.\ die Fachschaftsvertretung (FSV)}

\fibelabk{FSR}{Fachschaftsrat}

\fibelabk{FSV}{Fachschaftsvertretung}

\fibelabk{FT}{Festkörpertheorie; selten auch: Funktionentheorie}

\fibelabk{GAU}{Größter Anzunehmender Unfall; naja, und ein Super-GAU ist dann ein\dots}

\fibelabk{HFG}{Hochschulfreiheitsgesetz}

\fibelabk{HIS LSF}{Elektronisches Vorlesungsverzeichnis der Universität Münster. Dient der Übersicht und der vorläufigen Anmeldung zu Lehrveranstaltungen.\\
Nicht zu verwechseln mit QISPOS!! (siehe dort)}

\fibelabk{HS}{Hörsaal}

\fibelabk{HSP}{Hochschulsport; sehr günstige Angebote fast ALLER existierender Sportarten}

\fibelabk{i.A.}{im Allgemeinen, im Auftrag, in Arbeit u.v.m.}

\fibelabk{IG~1}{Institutsgruppe~1; Hauptgebäude der Physik; IG~2 gibt es nicht\dots\ (Größenwahn der 70er beendet!)}

\fibelabk{IVV}{Informationsverarbeitungs-Versorgungseinheit}

\fibelabk{JEF}{Junge Europäische Förderalisten, hochschulpolitische Gruppierung}

\fibelabk{Jovel}{Masematte für gut, ausgezeichnet, schön, etc. (und `ne Großraumdisco am Albersloherweg)}

\fibelabk{JuSo-HSG}{Jungsozialistenhochschulgruppe, hochschulpolitische Gruppierung}

\fibelabk{KFH}{Katholische Fachhochschule}

\fibelabk{KFWN}{Kommision für Forschung und wissenschaftlichen Nachwuchs}

\fibelabk{KLSA}{Kommission für Lehre und studentische Angelegenheiten}

\fibelabk{Koedukation}{gemeinsame Erziehung von Personen männlichen und weiblichen Geschlechts; soll am Physikfachbereich vielleicht auch einmal eingeführt werden}

\fibelabk{Kommilitone, Kommilitonin}{wurde als Anrede unter Studierenden gebraucht und bedeutet soviel wie Studienkollege; historisch: Waffengefährte; im neuzeitlichen wissenschaftlichen Betrieb abgelöst durch "lieber Kollege/liebe Kollegin"}

\fibelabk{KP}{Kernphysik}

\fibelabk{KÜ}{Kanalübergang; im Norden außerhalb Münsters gelegener Freizeit- und Badetreff}

\fibelabk{Kuhviertel}{Gebiet um die Kuhstraße; lokales Maximum der Kneipenkonzentration, dementsprechend sind dort größere Mengen an Studierenden und Studierten anzutreffen}

\fibelabk{LASER}{Light Amplification (by) Stimulated Emission (of) Radiation}

\fibelabk{Leeze}{Masematte für Fahrrad}

\fibelabk{LHG}{Liberale Hochschulgruppe, hochschulpolitische Gruppierung}

\fibelabk{Masematte}{Münsteraner "Slangsprache"}

\fibelabk{MFG}{Mitfahrgelegenheit, findet sich auf Aushängen oder bei der Mitfahrzentrale; manchmal auch: Mit freundlichen Grüßen}

\fibelabk{MP}{Materialphysik}

\fibelabk{MZ}{Münstersche~Zeitung}

\fibelabk{na dann\dots}{wöchentliche kostenlose Zeitschrift; Inhalt im wesentlichen: Kinoprogramm, Veranstaltungshinweise, Kleinanzeigen, Mensaplan; gibt's mittwochs an tausend und einer Stelle in Münster, z.B.\ in der Mensa}

\fibelabk{n.n.}{nomen nominandum; "noch nicht bekannt, wer es machen wird"}

\fibelabk{NWZ}{Naturwissenschaftliches Zentrum}

\fibelabk{NWZnet}{Rechnerumgebung der Fachbereiche Biologie, Chemie und Physik}

\fibelabk{o.B.d.A.}{"ohne Beschränkung der Allgemeinheit", Lieblingskürzel diverser Mathe- und Physikprofs, oft auch als oE ("ohne Einschränkung") abgekürzt}

\fibelabk{OE}{Orientierungseinheit für Erstsemester}

\fibelabk{Per Anhalter durch die Galaxis}{für Physikstudenten unbedingt erforderliches Werk der wissenschaftlichen Literatur}

\fibelabk{PI}{Physikalisches Institut}

\fibelabk{q.e.d.}{quod erat demonstrandum, lat.\ "was zu beweisen war"; bei fehlerhaften Beweisen auch scherzhaft als "quo errat demonstrator" (worin sich der Beweisende irrt) oder "quod est dubitandum" (was anzuzweifeln ist) gelesen}

\fibelabk{QISPOS}{wird zur Belegung von Lehrveranstaltungen verwendet. Alle Bachelor- und Master-Studierenden müssen ihre Lehrveranstaltungen hier belegen, um sich diese für ihr Studium anrechnen lassen zu können.\\
Nein, wir wissen auch nicht, was "QISPOS" bedeuten soll.}

\fibelabk{RCDS}{Ring christlich demokratischer Studenten, hochschulpolitische Gruppierung}

\fibelabk{Repetitorium}{geraffte und zielgerichtete Wiederholung des Vorlesungsstoffes als Vorbereitung auf Klausur oder Prüfung. Gibt es als Unterrichtsveranstaltung, wird aber auch freiberuflich gegen Entgelt angeboten. Die Unsitte der freiberuflichen Lehre gegen Bezahlung ist in der Physik nicht verbreitet.}

\fibelabk{Rieselfelder}{Naturschutzgebiet und Vogelreservat nordöstlich von Münster, sehr schönes Ausflugsziel}

\fibelabk{Rückmeldung}{ärgerliche Pflicht eines jeden Studierenden am Ende des Semesters; Zahlung der Semesterbeitrags, Übersicht der belegten Veranstaltungen}

\fibelabk{Schont}{Masematte für Toilette}

\fibelabk{schovel}{Masematte für schlecht, unfair, ätzend, gemein, sch\dots, usw.}

\fibelabk{SP/StuPa}{Studierendenparlament; Organ der studentischen Selbstverwaltung, setzt sich zusammen aus den von der Studierendenschaft gewählten studentischen Vertretern}

\fibelabk{SR}{Seminarraum; sieht gewöhnlich aus wie ein Klassenzimmer}

\fibelabk{SS / SoSe}{Sommersemester}

\fibelabk{s.t.}{sine tempore, also ohne Viertelstündchen}

\fibelabk{studium generale}{Ringvorlesung für Hörer/innen aller Fachbereiche unter einem Oberthema, das jeweils aus der Sicht der verschiedenen Wissenschaften beleuchtet wird; (in der Aula des Schlosses). Empfehlenswert!}

\fibelabk{SWS}{Eine Semester-Wochen-Stunde ist die Zeit, die ein Fach je Woche je Semester für Vorlesungen, Übungen, Praktika etc.\ in Anspruch nimmt. Steht da z.B.\ Physik~I, Vorlesung mit Übungen, 6+4 SWS, so hat man in einer Woche 6~Stunden Vorlesung + 4~Stunden Übung in Physik~I.}

\fibelabk{TP}{Theoretische Physik}

\fibelabk{trivial}{ganz einfache Sache, die eh keiner versteht und die deshalb auch nicht näher betrachtet wird}

\fibelabk{uFaFo}{unabhängiges Fachschaften-Forum, hochschulpolitische Gruppierung}

\fibelabk{ULB}{Universitäts- und Landesbibliothek}

\fibelabk{Verbindungen}{im letzten Jahrhundert geschaffene studentische Institution zur Einrichtung von Seilschaften in Industrie und Verwaltung; häufig nach archaischen Fecht- und Saufritualen.}

\fibelabk{V.i.S.d.P.}{Verantwortliche/r im Sinne des Presserechts}

\fibelabk{VV}{Vollversammlung; Veranstaltung, bei der sich alle Studierenden eines Fachbereiches versammeln, um akute Fragen zu diskutieren}

\fibelabk{WN}{Westfälische~Nachrichten (siehe auch MZ)}

\fibelabk{WS / WiSe}{Wintersemester}

\fibelabk{ZaPF}{Zusammenkunft aller Physik-Fachschaften}

\fibelabk{ZIV}{Zentrum für Informations-Verarbeitung}

\fibelabk{ZSB}{Zentrale Studienberatung, am Schloss}

\fibelsig{Raffi}
\end{multicols}
