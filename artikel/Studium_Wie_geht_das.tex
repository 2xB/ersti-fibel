% Autor: Simon May
% Datum: 2014-08-19
\section{Studium -- Wie geht das?}
% alternative Position für das Bild
%\begin{center}
%	\vspace{-0.5cm}
%	\includegraphics[width=0.7\linewidth]{res/bleistiftkauen.jpg}
%	\vspace{-0.5cm}
%\end{center}
\begin{multicols}{2}
% subsections in diesem Artikel kleiner (mit "\normalsize") darstellen,
% sonst ist zu wenig Platz
\addtokomafont{subsection}{\normalsize}
% etwas weniger Platz vor/nach subsections, sonst ist immer noch zu wenig Platz
\renewcommand{\fibelsubsectionpre}{\fibelsubsubsectionpre}
\renewcommand{\fibelsubsectionpost}{\fibelsubsubsectionpost}

\textbf{Wir fangen hier ganz langsam an. Für jene, die schon etwas mehr wissen, haben wir uns bemüht, viele kleine Zwischenüberschriften in den Text einzubauen; so könnt ihr auch mittendrin einsteigen oder später mal etwas nachschauen.}

\subsection*{Vorlesungen}
Vorlesungen sind die häufigsten Lehrveranstaltungen an Unis. In der Physik bzw.\ Mathematik schreibt der Professor meist die wichtigen Inhalte an Tafel/Beamer und man versucht möglichst schnell alles zu entziffern und vollständig mitzuschreiben. Wenn ihr Glück habt, stellt euer Professor Skizzen oder ein Skript (vollständige Mitschrift) ins Internet. In anderen Fächern ("Nebenfach"/Wahlpflichtfach: Informatik, Chemie, Geophysik, Philosophie, Mathematik, Psychologie, BWL, VWL) kann die Vorlesung auch aus einer PowerPoint-Präsentation bestehen. Die Qualität dieser Veranstaltungen ist also sehr stark vom Professor abhängig. Das Nacharbeiten der Vorlesungen ist meist notwendig, da sich, anders als in der Schule, der Hauptteil des Studiums zu Hause vollzieht. Auch wenn bis zu zweihundert Leute in einer Vorlesung sitzen, solltet ihr euch trauen, Zwischenfragen zu stellen.\\
Beim Besuch der Vorlesungen solltet ihr darauf Wert legen, den groben Überblick zu behalten.

\subsection*{Frust}
Aber Achtung, es ist am Anfang des Studiums durchaus üblich, auch den gröbsten Überblick völlig zu verlieren. Außerdem sind alle anderen viel besser als ihr -- glücklicherweise nur scheinbar. Fast jeder kennt diesen Frust am Anfang oder sogar im Mittelteil des Studiums. Das legt sich aber im Laufe der Zeit -- man braucht also nicht gleich aus dem Fenster zu springen. Manchmal hilft es, nicht sofort zu versuchen, alles zu verstehen, sondern stattdessen den Stoff nur schrittchenweise anzuschauen und nachzuvollziehen. Irgendwann kommt dann hoffentlich das Aha-Erlebnis ganz von alleine.

\subsection*{Aufgabenzettel}
Zu den Vorlesungen verteilen die Professoren oft (eigentlich jede Woche) Übungszettel mit Aufgaben zum Übungsstoff, die regelmäßig bearbeitet werden müssen, damit man zur Klausur zugelassen wird. Teilweise (z.\,B.\ Chemie) dienen sie nur zur Selbstkontrolle. Auch wenn die Aufgaben in Chemie nicht abgegeben werden müssen, empfiehlt sich unbedingt ihre Bearbeitung, da die Klausur in den meisten Fällen fast ausschließlich aus ähnlichen Aufgaben besteht. In anderen Nebenfächern ist dies zum Teil auch anders; die Bedingungen werden immer in der ersten Stunde bekannt gegeben.

\subsection*{Übungen}
Da die Besprechung der Aufgaben im Vorlesungsrahmen nicht möglich ist, gibt es eine weitere Art von Lehrveranstaltungen, die Übungen. Eine Übungsgruppe hat meist zwischen acht und 20~Teilnehmern und wird von einem höhersemestrigen Studierenden betreut. Die Einteilung der Gruppen erfolgt meist in einer der ersten Vorlesungen im Semester. Obwohl in den Übungen hauptsächlich die Aufgaben der Übungszettel gerechnet werden, sollen die Übungen auch zum Verständnis des übrigen Vorlesungsstoffes beitragen, was aber oft ein wenig zu kurz kommt.\\
Deshalb -- solltet ihr Fragen zur Vorlesung haben -- hier ist der Ort, sie zu stellen. Manchmal müsst ihr auch ein- oder zweimal Aufgaben (von den Übungszetteln) an der Tafel vorrechnen, um zur Klausur zugelassen zu werden. Deshalb wartet nicht so lange, sondern meldet euch lieber gleich am Anfang des Semesters, wenn die Aufgaben noch einfach sind.

\subsection*{Lerngruppen}
Es empfiehlt sich, die Aufgabenzettel in Gruppen von zwei bis vier Leuten zu bearbeiten und einen festen wöchentlichen Termin zu vereinbaren. Zum einen macht die Berechnung dann wesentlich mehr Spaß, zum anderen ist eine ständige gegenseitige Ergebnis- und Erfolgskontrolle gegeben. Zudem sind die Aufgaben oft so schwierig, dass man sie alleine kaum oder nur mit deutlich mehr Zeitaufwand lösen kann.

\subsection*{Praktika}
Praktika bestehen in der Physik meist darin, dass man an vollkommen fertigen Versuchsaufbauten mitunter einige Kabel umstöpseln darf, ein paar Messwerte aufnimmt und diese in ein Diagramm einträgt, um die obligatorische Ausgleichsgerade hindurch zu legen. Die wesentliche Arbeit besteht im theoretischen Vorbereiten der Versuche und im Schreiben der Versuchsprotokolle. Praktika finden als Blockpraktika (Chemie) an mehreren Tagen hintereinander (in den Semesterferien) oder wöchentlich statt. In der Physik beginnen die Praktika erst im dritten Semester.

\subsection*{Rückmelden}
Die sogenannte Rückmeldung findet jeweils in den letzten Wochen jeden Semesters statt, letzter Termin ist in der Regel der letzte Tag der Vorlesungszeit -- danach wird es teuer. Ihr bekommt alle Unterlagen zur Rückmeldung Mitte bis Ende der Vorlesungszeit per Mail zugesandt. Bei der Rückmeldung müsst ihr der Uni angeben, "dass es euch als Studentin bzw.\ Studenten noch gibt", und dass ihr auch im folgenden Semester noch da sein werdet. Dazu müsst ihr nur den Semesterbeitrag/Sozialbeitrag für das kommende Semester entrichten. Entweder geschieht dieses durch Überweisung mit dem zugeschickten Formular oder ihr erteilt der Uni eine Einzugsermächtigung.

\subsection*{Reinschnuppern in andere Fachbereiche}
Grundsätzlich seid ihr als Studentin bzw.\ Student
eines Fachbereichs berechtigt, an den Lehrveranstaltungen aller anderen Fachbereiche teilzunehmen und sogar Modulabschlussklausuren zu schreiben. Ausnahmen existieren nur bei zulassungsbeschränkten Veranstaltungen. Aber nach Absprache mit dem Leiter der Veranstaltung sind auch dann meistens Arrangements möglich. Interessant für Physiker könnten zum Beispiel wissenschaftstheoretische Seminare im Fachbereich Philosophie sein. Wenn man sich für andere Fachgebiete interessiert, sollte man sich schon in den Semesterferien die kommentierten Vorlesungsverzeichnisse anderer Fachbereiche holen, die es meist in der jeweiligen Fachschaft gibt, bzw.\ auch im Internet. Um als Studentin bzw.\ Student unseres Fachbereichs über den Tellerrand der Physik hinaus zuschauen, bietet sich das zweite Semester an. Man hat dann schon ein wenig den Überblick gewonnen und hat andererseits in diesem Semester relativ wenig zu tun.

\fibelsig{Andreas G.}
\end{multicols}
