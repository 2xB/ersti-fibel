% Autor: Simon May
% Datum: 2014-09-17

\enlargethispage{\baselineskip}
\vspace*{-1cm}
\section{Das große Bluffspiel}
\vspace*{-0.3cm}
\begin{pullquote}{shape=image, image=res/bluff_chips_karten.jpg, imageopts={width=3.3cm}}
% LaTeX-Warnungen vermeiden
\hbadness=5000
An dieser Stelle möchte ich einmal ein paar Worte über eine weit verbreitete Unart an der Hochschule verlieren: "Den großen Bluff". Was um alles in der Welt ist das? Nun, du wirst es in den Vorlesungen und Übungen erleben: Da sitzen um dich herum lauter hochintelligente und urgescheite Leute, die aufmerksam dem Professor lauschen und all das verstehen, woran du selber fast verzweifelst. Ähnlich wie beim Poker: Es gewinnt derjenige, der das coolste und undurchschaubarste Gesicht macht, obwohl er gar nichts auf der Hand hat (bzw.\ in diesem Fall: Obwohl er kein Wort verstanden hat).\pullquotenl

Manche Studierende beherrschen das "Bluffspiel" bis zur Vollendung. Das kann einen gerade am Anfang ganz schön fertigmachen, wenn man den Eindruck hat, dass alle anderen den Stoff schon längst kapiert haben, und dass doch eigentlich alles ganz trivial sein muss. Du willst dir keine Blöße geben und fängst an, bei dem Spiel mitzuspielen und nach kurzer Zeit beherrschst du es auch, doch hilft dies leider keinem. Nach einer gewissen Zeit studieren dann alle alleine nebeneinander her -- keiner traut sich, andere um Hilfe zu bitten, denn jeder denkt für sich: "Es könnte ja jemand merken, dass ich nicht viel weiß." Am Ende tun sich so schließlich alle schwerer, als es notwendig ist.\pullquotenl

Wenn es einmal so weit gekommen ist, ist es unwahrscheinlich schwierig, mit dem "Bluffspiel" wieder aufzuhören. Dabei könnte es viel einfacher sein, wenn jeder ein bisschen auf den anderen zugehen würde. Die selbstsichere Maske der Kommilitonen ist nämlich meistens nur aufgesetzt, um die eigene Unsicherheit zu vertuschen. Wenn man die Distanz einmal überwunden hat, wird man feststellen, dass es den anderen genauso geht, wie einem selbst. Meist sind sie dann ganz froh, dass jemand auf sie zugekommen ist, was sie selbst vielleicht nicht geschafft haben und sie sich nun einmal so zeigen, wie sie wirklich sind. Das Studium ist wirklich viel leichter zu bewältigen, wenn man Freunde hat, mit denen man lachen, weinen, lernen oder über Probleme sprechen kann. Wenn alle etwas mehr zusammenhalten, macht es auch wesentlich mehr Spaß. Deshalb bitte ich dich: Versuche nicht, dass Studium als Einzelkämpfer anzugehen und den anderen vorzuspielen, dass du keine Hilfe brauchst. Man braucht kein Held zu sein, um akzeptiert zu werden.

\fibelsig{Andreas G., Simon}
\end{pullquote}
