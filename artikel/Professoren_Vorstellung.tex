% Autor: Simon May
% Datum: 2014-09-09
\section{Eure Professoren stellen sich vor}
\vspace{-0.5cm}
\textbf{Auf den folgenden zwei Seiten stellen sich eure beiden Professoren vor. Sie werden gemeinsam die "Physik~1" bis "Physik~3" lesen. Prof.\ Thiele wird sich dabei um die theoretischen und Prof.\ Demokritov um die experimentellen Aspekte des Studiums kümmern. Zudem wird Prof.\ Werner die Vorlesungen "Mathematik für Physiker" lesen. Da diese drei Professoren euch also eine Zeit lang begleiten werden, ist es doch mal interessant zu wissen, was sie gemacht haben, bevor sie an die Uni Münster kamen, und wie ihre aktuelle Forschung aussieht.}

\begin{multicols}{2}
\begin{center}
\includegraphics[width=0.9\columnwidth]{res/vorstellungsfotos/demokritov.jpg}\\
\smallskip
Prof.\ Dr.\ Sergej O.\ Demokritov\\
Institut für Angewandte Physik
\end{center}

In den nächsten 20~Monaten werde ich Sie, zusammen mit meinem Kollegen Prof.\ Dr.\ U.\ Thiele, durch den "Integrierten Kurs" (Physik~1--3) führen. Da ich bereits seit 2004 an der Westfälischen Wilhelms-Universität unterrichte und forsche, erlebe ich dieses Abenteuer mit Ihnen nun zum zweiten Mal. Interessanterweise hat die meistzitierte Aussage meiner Vorlesung von damals, dass man kostenlosen Käse nur in einer Mausefalle finde, mit der Physik nicht viel zu tun. Mein Anliegen ist, dass Sie hier nicht nur die Grundlagen der Physik verstehen, sondern auch lernen, wie man lernt. Mit dieser Erfahrung wird das gesamte Studium einfacher für Sie. 

\begin{center}
\includegraphics[width=0.95\columnwidth]{res/mausefalle.jpg}\\
\hfill {\textcopyright} Paul Turton
\end{center}

Ich bin an der WWU im Institut für Angewandte Physik tätig. Dort befassen wir uns mit Quantum-Nanomagnetismus, besonders mit dynamischen und thermodynamischen Phänomenen in magnetischen Nanosystemen. Wir haben maßstäblich dazu beigetragen, dass das Forschungsfeld unter dem Namen "Magnonik" weltweit floriert. Entsprechend gibt es am Fachbereich im Rahmen des Master-Studiengangs Physik eine Vertiefungsrichtung "Photonik und Magnonik". Ich hoffe, dass auch Sie dieses Themengebiet interessant finden und freue mich, Sie später in meiner Arbeitsgruppe als Bachelor- oder Masterstudent begrüßen zu dürfen.

Ich habe in der USSR (von 1976 bis 1982) Physik studiert und dort auch (1987) im Institut des berühmten russischen Physikers Pjotr Kapiza (Nobelpreis in der Physik 1978) promoviert. Im Jahr 1990 bin ich als Alexander von Humboldt-Stipendiat nach Deutschland gekommen. Zuerst arbeitete ich am FZ Jülich in der Gruppe von Prof.\ Dr.\ P.\ Grünberg -- 2007 wurde ihm für seine Forschung der magnetischen Nanostrukturen der Nobelpreis für Physik zuerkannt. Anschließend war ich als Habilitand an der TU Kaiserslautern (1995--2004) tätig, bevor ich an die WWU berufen wurde.

Man kann wohl sagen, dass ich in meinem Werdegang die besten Seiten der russischen und deutschen Forschung und Physik-Hochschulausbildung kennengelernt habe. Diese Erfahrung werde ich gerne während der nächsten drei Semester mit Ihnen teilen. Ich freue mich darauf und wünsche Ihnen einen erfolgreichen Start in das Studium der Physik.
\end{multicols}

\newpage

\begin{multicols}{2}
\begin{center}
\includegraphics[width=0.71\columnwidth]{res/vorstellungsfotos/thiele_cropped_scaled.jpg}\\
Prof.\ Dr.\ Uwe Thiele\\
Institut für Theoretische Physik
\end{center}

Wir werden uns die nächsten Semester regelmäßig in den Vorlesungen Physik~1 bis~3 sehen. Ich hoffe, dass Sie alle sich erfolgreich den Stoff erarbeiten werden, den Prof.\ Demokritov (Experiment) und ich (Theorie) vorstellen auf unserem Weg durch Mechanik, Thermodynamik, Elektrodynamik, Optik und Spezielle Relativitätstheorie. Im Folgenden möchte ich mich und meine Arbeitsgruppe kurz bei Ihnen vorstellen.

Ich habe um 1990 herum in Dresden Physik studiert, um herauszufinden, "was die Welt im Innersten zusammenhält". Nachdem ich konsequenterweise mit Elementarteilchenphysik geliebäugelt hatte, ließ ich mich dann für meine Promotion von der Physik komplexer Systeme anziehen. Arbeitsaufenthalte führten mich u.a.\ nach Augsburg, Bayreuth, Berkeley, Dresden und Madrid, bevor ich ab 2007 in Loughborough~(UK) lehrte und forschte.

Im zeitigen Frühjahr 2014 verlagerte ich diese Aktivitäten nach Münster an das Institut für Theoretische Physik, wo wir in der Gruppe \textit{Selbstorganisation und Komplexität} universelle Eigenschaften komplexer Nichtgleichgewichtssysteme mit theoretischen und numerischen Methoden erforschen. Diese Systeme bestehen oft aus vielen mikroskopischen, nichtlinear wechselwirkenden Komponenten, was außerhalb (und fern) vom Gleichgewicht zur spontanen Entwicklung von Strukturen führt, die nicht von außen aufgeprägt werden, sondern durch Selbstorganisationsprozesse entstehen. Im alltäglichen Leben kommen solche Phänomene in vielfacher Weise vor, z.B.\ als Konvektion im Milchkaffee, bei der Ausbildung (und Dynamik) von Tierfellmustern, Wasserwellen, Sanddünen oder Wolkenbändern.

\begin{center}
\includegraphics[width=0.7\columnwidth]{res/thiele_strukturen.png}\\
{\footnotesize % Beispiele berechneter Strukturen: Pilzstruktur bei thermischer Konvektion (links), Fingerinstabilität beim Entnetzen und Verdunsten einer Nanoteilchensuspension (unten rechts) und Streifenmuster beim Transfer oberflächenaktiver Substanzen von einem Bad auf eine bewegte Platte (oben rechts).
Bilder von J.\ Lülff, A.\ Archer, M.\ Wilczek.}
\end{center}

Ein Schwerpunkt der Gruppe ist weiche und aktive Materie, deren Dynamik oft grenzflächendominiert ist, d.h.\ sie wird durch Grenzflächenenergien kontrolliert. Beispiele sind Tropfen auf Substraten, Schichten von Flüssigkristallen und kolloidalen Suspensionen. Ein wichtiges Ziel ist das Verständnis der strukturbildenden Wechselwirkungen diverser voneinander abhängiger Transportprozesse und Phasenübergänge.

Ähnliche Fragen treten auch bei der Dynamik biologischer Zellen, dem Gewebewachstum sowie bei der Bewegung von Schwärmen auf. Von großem Interesse ist auch die Strukturbildung durch Selbstassemblierung mikroskopischer Bestandteile. Deren Kontrolle wird benutzt, um großflächig Oberflächen mit wohlstrukturierten Schichten zu bedecken. Wir möchten verstehen, wie die grundlegenden Eigenschaften der Materialien und Prozesse zur Bildung spezieller funktionaler Muster führen und man diese durch aufgeprägte Felder kontrollieren kann.

Außerdem beschäftigen wir uns mit thermischer Konvektion (durch Temperaturgradienten angetriebene Strömungen) und Turbulenz. Fast alle solche Flüsse in der Natur (z.B.\ Atmosphäre, Ozeane, Plattentektonik) sind turbulent und aufgrund ihres chaotischen Verhaltens schwer zu behandeln. Insbesondere sind genaue Vorhersagen über längere Zeiten fast unmöglich (siehe Wetterbericht\dots). Stattdessen ist unser Ziel, neue kombinierte statistische und numerische Beschreibungen zu entwickeln, die dann auf Supercomputern numerisch gelöst werden und zu einem besseren Verständnis turbulenter Systeme führen.

Obwohl in der Beschreibung nicht leicht erkennbar, möchte ich Ihnen abschließend versichern, dass der Weg zum Verständnis solch komplexer Phänomene in den Grundvorlesungen beginnt, die wir demnächst gemeinsam bestreiten. Dafür wünsche ich Ihnen viel Erfolg. Sollten Sie mehr über komplexe Systeme erfahren wollen, sind Sie herzlich eingeladen, die Arbeitsgruppe zu besuchen.
\end{multicols}

\newpage

\begin{multicols}{2}
\begin{center}
\includegraphics[width=0.9\columnwidth]{res/vorstellungsfotos/werner.jpg}\\
Apl.\ Prof.\ Dr.\ Wend Werner\\
Mathematisches Institut
\end{center}

Mir werden Sie in der nächsten Zeit in den Vorlesungen zur "Mathematik für Physiker" begegnen. Mein Studium von Mathematik und Physik habe ich an der Freien Universität Berlin absolviert; Diplom und Promotion habe ich auch jeweils dort abgeschlossen. Meine Habilitation habe ich an der Universität Paderborn gemacht und bin nun seit gut einem Jahrzehnt Hochschullehrer in Münster.

\[
\resizebox{0.45\hsize}{!}{$\displaystyle\sum_{n = 1}^\infty \frac{1}{n^2} = \frac{\pi^2}{6}$}
\]

Die Themen von Promotion und Habilitation betrafen geometrische Fragen in Räumen unendlicher Dimension. In letzter Zeit war das vor allem "Nichtkommutative Geometrie", ein Gebiet, welches versucht, einen mathematischen Formalismus zu finden, der in der Lage ist, Quanten- und relativistische Physik in einheitlicher Weise zu beschreiben. Wie so oft beim Zusammenspiel von Mathematik und Physik sind auch hier interessante, rein mathematische Fragestellungen in Erscheinung getreten.

\begin{center}
\includegraphics[width=0.85\columnwidth]{res/comics/calvin_mathe.png}
\end{center}

Der Zyklus "Mathematik für Physiker" ist eine kleine Herausforderung, da in vergleichsweise kurzer Zeit eine größere Stoffmenge vermittelt werden muss, die nichtsdestotrotz von den Teilnehmern anschließend handwerklich beherrscht werden muss.

Aber, keine Angst: Wir werden sehr langsam beginnen und erst im Laufe der Zeit Fahrt aufnehmen.
\end{multicols}

\medskip

\begin{center}
\includegraphics[width=0.9\textwidth]{res/comics/xkcd_purity.png}
\end{center}

