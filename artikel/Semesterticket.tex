% Autor: Simon May
% Datum: 2014-08-12
\section{Das Semesterticket, oder die neue Freiheit}

\begin{center}
\includegraphicscompressed[width=0.8\textwidth]{res/bus_bahn.png}
\end{center}
\begin{multicols}{2}
\begin{figure*}[t]
\subsection*{Geltungsbereich des NRW-Tickets}
\includegraphics[width=\textwidth]{res/regionalverkehrsplan_2012.png}
\end{figure*}
\textbf{Früher hab' ich mal geglaubt, dass Semesterticket sei nur für den Weg nach Hause bestimmt, aber da lag ich nicht ganz richtig -- das Semesterticket kann noch viel mehr.}

\includegraphicscompressed[width=\columnwidth]{res/semesterticket.png}

Wenn du ordentlich bei der Westfälischen~Wilhelms-Universität~Münster immatrikuliert bist und deinen Studentenbeitrag bezahlt hast, bekommst du ein freundliches Schreiben, dass du für das kommende Semester gemeldet bist. In diesem Schreiben befinden sich auch einige Semesterbescheinigungen (in schönem Blau) und dein VGM~Semesterticket (inklusive der "NRW-Ticket"-Option). Das Semesterticket musst du aus dem Brief herausnehmen und brav zusammen mit einem gültigen Lichtbildausweis vorzeigen, wenn du dich mit Bus und Bahn fortbewegen willst. Ohne einen gültigen Lichtbildausweis ist das Semesterticket nicht gültig. Die meisten Busfahrer sehen es nicht als Verstoß an, wenn man mal keinen Lichtbildausweis bei sich trägt, doch spätestens, wenn du das Semesterticket für Fahrten mit der Bahn verwendest, musst du einen gültigen Ausweis bei dir tragen.

Der Universität~Münster ist es nach langen Diskussionen und einer Urabstimmung gelungen, mit der Deutschen~Bahn einen Vertrag über ein NRW-Ticket abzuschließen. Daher darfst du dich im kompletten Nahverkehr in NRW bewegen. Zusätzlich gilt das Ticket auch für die Bahn-Strecke von Münster aus nach Osnabrück (und von da aus auch nach Bielefeld, Rheine und Halen), von Löhne über Hameln nach Altenbeken, von Bonn nach Siegen über Au (Sieg) und für die Strecke nach Enschede (Niederlande).

Nahverkehr bedeutet, dass du alle Busse, Straßenbahnen, U-Bahnen, S-Bahnen sowie alle Regionalbahnen -- RegionalExpress (RE), RegionalBahn (RB), und alle privaten Bahnen -- im Raum Nordrhein-Westfalen benutzen darfst. Natürlich zählt der Fernverkehr -- InterCity (IC) und InterCityExpress (ICE) -- nicht zum Nahverkehr.

% XXX Stimmt das noch?
Am Schloss (das große Gebäude, in dem sich unter anderem auch das Studierendensekretariat befindet) hast du schon so eine praktische "Plastikhülle" bekommen, die ideal für die Kombination von deinem Semesterticket und deinem Lichtbildausweis geeignet ist. Wenn du diese Hülle noch nicht bekommen hast, geh' schnell zum Schloss und hol' dir deine, sie liegen dort immer frei auf den Tischen.

Solltest du mal mit der Bahn unterwegs sein und dein Semesterticket liegt zuhause, ist das zwar ärgerlich, aber kein Beinbruch. Zunächst wirst du vom Schaffner als "Schwarzfahrer" aufgeschrieben, mit dem Vorbehalt einer \SI{40}{\euro}-Strafe. Dieses Schreiben musst du dann zusammen mit deinem Semesterticket am Schalter in Münster vorzeigen und musst nur eine Bearbeitungsgebühr in Höhe von ungefähr \SI{10}{\euro} bezahlen. Ärgerlich, aber besser, als die ganzen \SI{40}{\euro} zu zahlen.

Solltest du mal per Anschlusszug über die Grenzen des NRW-/Semestertickets hinaus fahren wollen, denk daran, dass nötige Ticket rechtzeitig (am besten gleich als "Viererticket") zu lösen, denn Nachlösen im Zug ist speziell für Studenten der Universität~Münster nicht erlaubt! Warum das so ist, konnte mir bis dato keiner erklären.
\columnbreak

Es ist von Zeit zu Zeit ganz ratsam, sich über die Optionen deines Semestertickets neu zu informieren. So wurde zum Beispiel vor einiger Zeit geändert, dass du in Bussen \emph{im Stadtgebiet Münster} ab 19~Uhr sowie an Wochenenden und Feiertagen ganztägig eine weitere Person oder ein Fahrrad mitnehmen darfst (gilt nicht für Bahnverbindungen). Aufgrund des begrenzten Platzangebotes entscheidet der Busfahrer, ob ein Fahrrad mitgenommen werden darf oder nicht.

\begin{center}
\subsection*{Aktualisierte Infos gibt es immer unter}

\url{https://www.stadtwerke-muenster.de}

(genauere Links findet ihr im DPÜ auf Seite \pageref{dpü} dieser Ersti-Fibel)
\end{center}

\fibelsig{Andreas G.}
\end{multicols}
